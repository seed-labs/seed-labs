\documentclass[11pt]{article}

\usepackage[most]{tcolorbox}
\usepackage{times}
\usepackage{epsf}
\usepackage{epsfig}
\usepackage{amsmath, alltt, amssymb, xspace}
\usepackage{wrapfig}
\usepackage{fancyhdr}
\usepackage{url}
\usepackage{verbatim}
\usepackage{fancyvrb}
\usepackage{adjustbox}
\usepackage{listings}
\usepackage{color}
\usepackage{subfigure}
\usepackage{cite}
\usepackage{sidecap}
\usepackage{pifont}
\usepackage{mdframed}
\usepackage{textcomp}
\usepackage{enumitem}


% Horizontal alignment
\topmargin      -0.50in  % distance to headers
\oddsidemargin  0.0in
\evensidemargin 0.0in
\textwidth      6.5in
\textheight     8.9in 

\newcommand{\todo}[1]{
\vspace{0.1in}
\fbox{\parbox{6in}{TODO: #1}}
\vspace{0.1in}
}


\newcommand{\unix}{{\tt Unix}\xspace}
\newcommand{\linux}{{\tt Linux}\xspace}
\newcommand{\minix}{{\tt Minix}\xspace}
\newcommand{\ubuntu}{{\tt Ubuntu}\xspace}
\newcommand{\setuid}{{\tt Set-UID}\xspace}
\newcommand{\openssl} {\texttt{openssl}}


\pagestyle{fancy}
\lhead{\bfseries SEED Labs}
\chead{}
\rhead{\small \thepage}
\lfoot{}
\cfoot{}
\rfoot{}


\definecolor{dkgreen}{rgb}{0,0.6,0}
\definecolor{gray}{rgb}{0.5,0.5,0.5}
\definecolor{mauve}{rgb}{0.58,0,0.82}
\definecolor{lightgray}{gray}{0.90}


\lstset{%
  frame=none,
  language=,
  backgroundcolor=\color{lightgray},
  aboveskip=3mm,
  belowskip=3mm,
  showstringspaces=false,
%  columns=flexible,
  basicstyle={\small\ttfamily},
  numbers=none,
  numberstyle=\tiny\color{gray},
  keywordstyle=\color{blue},
  commentstyle=\color{dkgreen},
  stringstyle=\color{mauve},
  breaklines=true,
  breakatwhitespace=true,
  tabsize=3,
  columns=fullflexible,
  keepspaces=true,
  escapeinside={(*@}{@*)}
}

\newcommand{\newnote}[1]{
\vspace{0.1in}
\noindent
\fbox{\parbox{1.0\textwidth}{\textbf{Note:} #1}}
%\vspace{0.1in}
}


%% Submission
\newcommand{\seedsubmission}{You need to submit a detailed lab report, with screenshots,
to describe what you have done and what you have observed.
You also need to provide explanation
to the observations that are interesting or surprising.
Please also list the important code snippets followed by
explanation. Simply attaching code without any explanation will not
receive credits.}

%% Book
\newcommand{\seedbook}{\textit{Computer \& Internet Security: A Hands-on Approach}, 2nd
Edition, by Wenliang Du. See details at \url{https://www.handsonsecurity.net}.}

%% Videos
\newcommand{\seedisvideo}{\textit{Internet Security: A Hands-on Approach},
by Wenliang Du. See details at \url{https://www.handsonsecurity.net/video.html}.}

\newcommand{\seedcsvideo}{\textit{Computer Security: A Hands-on Approach},
by Wenliang Du. See details at \url{https://www.handsonsecurity.net/video.html}.}

%% Lab Environment
\newcommand{\seedenvironment}{This lab has been tested on our pre-built
Ubuntu 16.04 VM, which can be downloaded from the SEED website.}






\newcommand{\seedlabcopyright}[1]{
\vspace{0.1in}
\fbox{\parbox{6in}{\small Copyright \copyright\ {#1}\ \ by Wenliang Du.\\
      This work is licensed under a Creative Commons
      Attribution-NonCommercial-ShareAlike 4.0 International License.
      If you remix, transform, or build upon the material, 
      this copyright notice must be left intact, or reproduced in a way that is reasonable to
      the medium in which the work is being re-published.}}
\vspace{0.1in}
}






\lhead{\bfseries SEED Labs -- IP/ICMP Attacks Lab}
\newcommand{\ipFigs}{./Figs}


\begin{document}



\begin{center}
{\LARGE IP/ICMP Attacks Lab}
\end{center}

\seedlabcopyright{2020}


% *******************************************
% SECTION
% ******************************************* 
\section{Overview}

The objective of this lab is for students to gain the first-hand experience 
on various attacks at the IP layer. Some of the attacks may not work anymore,
but their underlying techniques are quite generic, and it
is important for students to learn these attacking techniques, so when they design
or analyze network protocols, they are aware of what attackers can do to protocols.
Moreover, due to the complexity of IP fragmentation, spoofing fragmented IP packets
is non-trivial.  Constructing spoofed 
IP fragments is a good practice for students to hone their
packet spoofing skills, which are essential in network security.  
We will use Scapy to conduct packet spoofing.  
This lab covers the following topics:

\begin{itemize}[noitemsep]
\item The IP and ICMP protocols
\item IP Fragmentation and the related attacks
\item ICMP redirect attack
\item Routing and reverse path filtering 
\end{itemize}



\paragraph{Videos.}
Detailed coverage of the IP protocol and the attacks at the IP layer can be found 
in the following:

\begin{itemize}
\item Section 4 of the SEED Lecture, \seedisvideo
\end{itemize}


\paragraph{Lab environment.} \seedenvironment


% *******************************************
% SECTION
% ******************************************* 
\section{Tasks 1: IP Fragmentation}

Two VMs are needed for this task. They should be connected to the same network, so they
can communicate with each other.

% -------------------------------------------
% SUBSECTION
% ------------------------------------------- 
\subsection{Task 1.a: Conducting IP Fragmentation}

In this task, students need to construct a UDP packet and send it to a UDP 
server. They can use \texttt{"nc -lu 9090"} to start a UDP server. 
Instead of building one single IP packet, students need to 
divide the packet into 3 fragments, each containing 32 bytes of data (the
first fragment contains 8 bytes of the UDP header plus 32 bytes of data).
If everything is done correctly, the server will display
96 bytes of data in total.  
The following is a sample code for constructing the first fragment.

\begin{lstlisting}
#!/usr/bin/python3
from scapy.all import *

# Construct IP header
ip = IP(src="1.2.3.4", dst="10.0.0.15")
ip.id    = 1000  # Identification
ip.frag  = 0     # Offset of this IP fragment
ip.flags = 1     # Flags

# Construct UDP header
udp = UDP(sport=7070, dport=9090)
udp.len  = 200   # This should be the combined length of all fragments

# Construct payload
payload = 'A' * 80    # Put 80 bytes in the first fragment

# Construct the entire packet and send it out
pkt = ip/udp/payload  # For other fragments, we should use ip/payload
pkt[UDP].checksum = 0 # Set the checksum field to zero
send(pkt, verbose=0)
\end{lstlisting}


It should be noted that the UDP checksum field needs to be set 
correctly. If we do not set this field, Scapy will calculate 
the checksum for us, but this checksum will only be based on the 
data in the first fragment, which is incorrect.
If we set the checksum field to  zero, Scapy will leave it alone.
Moreover, the recipient will not validate the UDP checksum 
if it sees a zero in the checksum field, 
because in UDP, checksum validation is optional.


If you use Wireshark to observe traffic, it should also be noted that by default, 
Wireshark will reassemble fragments in the last fragment packet and
show it as a complete IP/UDP packet. To change that behavior,
we should disable IP fragment reassembly in Wireshark preferences.
Click the following menu sequence: \texttt{Edit} $\rightarrow$ \texttt{Preferences}; 
click the \texttt{Protocols} dropdown menu, find and click \texttt{IPv4}.
Uncheck the \texttt{"Reassemble fragmented IPv4 datagrams"} option. 






% -------------------------------------------
% SUBSECTION
% ------------------------------------------- 
\subsection{Task 1.b: IP Fragments with Overlapping Contents}

Similar to Task 1.a, students also need to construct 3 fragments to send data to a UDP server.
The size of each fragment is up to students.  The objective of this task is to create
overlapping fragments.  In particular, the first two fragments should overlap.  Please use
experiments to show what will happen when the overlapping occurs. Please
try the following overlapping scenarios separately:
 
 \begin{itemize} 
 \item The end of the first fragment and the beginning of the second
 fragment should have \texttt{K} bytes of overlapping, i.e., the last  
 \texttt{K} bytes of data in the first fragment should have the same
 offsets as the first \texttt{K} bytes of data in the second fragment. 
 The value of \texttt{K} is decided by students (\texttt{K} should be 
 greater than zero and smaller than the size of either fragment). In the reports, students
 should indicate what their \texttt{K} values are. 


 \item The second fragment is completely enclosed in the first fragment.
 The size of the second fragment must be smaller than the 
 first fragment (they cannot be equal).

 \end{itemize} 


Please try two different orders: (1) sending the first fragment first, and 
(2) sending the second fragment first. Please report whether the results will
be the same. 





% -------------------------------------------
% SUBSECTION
% ------------------------------------------- 
\subsection{Task 1.c: Sending a Super-Large Packet}

As we know, the maximal size for an IP packet is $2^{16}$ octets, because
the length field in the IP header has only 16 bits. 
However,
using the IP fragmentation, we can create an IP packet that 
exceeds this limit. Please construct such a packet, send
it to the UDP server, and see how the server responds to this 
situation. Please report your observation. 
  



% -------------------------------------------
% SUBSECTION
% ------------------------------------------- 
\subsection{Task 1.d: Sending Incomplete IP Packet}


In this task, we are going to use Machine A to launch a Denial-of-Service attack 
on Machine B. In the attack, Machine A sends  a lot of 
incomplete IP packets to B, i.e., these packets consist of 
IP fragments, but some fragments are missing. All these incomplete IP packets 
will stay in the kernel, until they time out. Potentially, this can
cause the kernel to commit a lot of kernel memory. In the past, this 
had resulted in denial-of-service attacks on the server. Please try 
this attack and describe your observation. 




% *******************************************
% SECTION
% ******************************************* 
\section{Task 2: ICMP Redirect Attack}

An ICMP redirect is an error message sent by a router to the sender of an
IP packet. Redirects are used when a router believes a packet is being
routed incorrectly, and it would like to inform the sender that it should
use a different router for the subsequent packets sent to that same
destination.


In our VM, there is a countermeasure against the ICMP redirect attack. Before doing
the task, we need to turn off the countermeasure by configuring the operating system
to accept ICMP redirect messages. 

\begin{lstlisting}
$ sudo sysctl net.ipv4.conf.all.accept_redirects=1
\end{lstlisting}

For this task, we should have two VMs, the victim VM (Host A) and 
the attacker VM (Host M). Students should pick a destination B, which
should be a host outside of our local network (e.g., an outside web server). 
Normally, when A sends a packet to B, the packet will go to the 
router provided by VirtualBox (usually it is \texttt{10.0.2.1} if we use
the default IP prefix for NAT Network). 

The objective of this task is to launch an ICMP redirect attack on Host A 
from Host M, such that when Host A sends packets to B, 
it will use M as the router, and hence sends those packets to M. 
Since M is controlled by the attacker, the attacker can 
intercept the packets, make changes, and then send 
the modified packets out. This is a form of the Man-In-The-Middle (MITM) attack. 
For the simplicity of this lab, students are not required to 
conduct the MITM part; they only need to demonstrate that 
their ICMP redirect attacks can successfully redirect
packets from A to B.


\paragraph{Code skeleton.} A code skeleton is provided in the following, with
some of the essential parameters left out. Students should fill in the proper 
values in the places marked by \texttt{@@@@}.  


\begin{lstlisting}
#!/usr/bin/python3

from scapy.all import *

ip = IP(src = @@@@,  dst = @@@@)
icmp = ICMP(type=@@@@, code=@@@@)
icmp.gw = @@@@

# The enclosed IP packet should be the one that 
# triggers the redirect message. 
ip2 = IP(src = @@@@, dst = @@@@)
send(ip/icmp/ip2/UDP());
\end{lstlisting}
 

\paragraph{Verification.}
To verify whether the ICMP redirect attack is successful, we can use 
the \texttt{"ip route get"} command to see what router will be used for a 
packet destination. For example, if we want to know
what router will be used for packets going to \texttt{8.8.8.8}, we can use 
the following command: 

\begin{lstlisting}
$ ip route get 8.8.8.8
8.8.8.8 via 10.0.2.1 dev enp0s3  src 10.0.2.4
    cache
\end{lstlisting}


\paragraph{Questions.} Please conduct the following experiments, and explain your observations:

\begin{enumerate}
\item Can you use ICMP redirect attacks to redirect to a remote machine? Namely,
the IP address assigned to \texttt{icmp.gw} is a computer not on the local LAN. 
Please show your experiment result, and explain your observation.  

\item Can you use ICMP redirect attacks to redirect to a non-existing machine on
the same network? Namely, the IP address assigned to \texttt{icmp.gw} is a local computer that
is either offline or non-existing. 
Please show your experiment result, and explain your observation.  
\end{enumerate}





% *******************************************
% SECTION
% ******************************************* 
\section{Task 3: Routing and Reverse Path Filtering} 


The objective of this task is two-fold: get familiar with routing, 
and understand a spoof-prevention mechanism 
called reverse path filtering, which prevents outside from spoofing 




% -------------------------------------------
% SUBSECTION
% ------------------------------------------- 
\subsection{Task 3.a: Network Setup}
 



\begin{figure}[htb]
\begin{center}
\includegraphics[width=0.8\textwidth]{\ipFigs/NetworkSetup.pdf}
\end{center}
\caption{Network setup}
\label{ip:fig:reverse_filtering}
\end{figure}

In this task, we will set up two networks and three VMs, \texttt{A}, \texttt{B}, and 
\texttt{R}. The network setup is depicted in Figure~\ref{ip:fig:reverse_filtering}. 
The IP addresses in the figure are only for illustration purposes; students can choose
different IP addresses in their setup.


The \texttt{10.0.2.0/24} network should use the \texttt{"NAT Network"} adaptor 
in VirtualBox. Both Machines \texttt{A} and \texttt{R} are attached 
to this network, so they can directly communicate with each other. 
The second network \texttt{192.168.60.0/24} should use the 
\texttt{"Internal Network"} adaptor in VirtualBox. Only
Machines \texttt{R} and \texttt{B} are attached to this network. 
Therefore, Machines \texttt{A} and \texttt{B} cannot directly 
communicate with each other.
The figure shows three machines inside the \texttt{192.168.60.0/24} network, but 
students only need to include one VM in their setup. 


Since Machine \texttt{R} is attached to both networks, it needs to 
have two network adaptors, one using the \texttt{"NAT Network"} type
and the other using the \texttt{"Internal Network"}. In VirtualBox, 
by default, only one network adaptor is enabled. We can
go to the VM Settings to enable the second one, but  we need to 
shutdown the VM first; otherwise, we will find out that
the \texttt{"Adaptor 2"} button is grayed out.


\paragraph{Set up static IP addresses.}
VMs attached to the \texttt{"NAT Network"} network will automatically get their 
IP addresses from the \texttt{DHCP} server, but for 
VMs on the \texttt{"Internal Network"}, VirtualBox does not provide DHCP, so the
VM must be statically configured. To do this, click the network icon on the top-right corner
of the desktop, and select \texttt{"Edit Connections"}. You will see a list
of \texttt{"Wired connections"}, one for each of the network adaptors used by the VM.
For Machine \texttt{B}, there is only one connection, but 
for Machine \texttt{R}, we will see two. To make sure
that you pick the one attached to the \texttt{"Internal Network"} adapter,
You can check the MAC address displayed in the pop-up window after you have
picked a connection to edit.
Compare this MAC address with the one displayed by the \texttt{ifconfig} command,
and you will know whether you have picked the right connection or not.

After selecting the right connection to edit,
pick the \texttt{"ipv4 Settings"} tab and select the
\texttt{"Manual"} method, instead of the default \texttt{"Automatic (DHCP)"}. Click
the \texttt{"Add"} button to set up the new IP address for the VM. See
Figure~\ref{ip:fig:internalnetwork} for details.


\begin{figure}[htb]
\begin{center}
\includegraphics[width=0.8\textwidth]{\ipFigs/InternalNetwork.pdf}
\end{center}
\caption{Manually set up the IP address for the \texttt{"Internal Network"} adaptor.}
\label{ip:fig:internalnetwork}
\end{figure}




% -------------------------------------------
% SUBSECTION
% ------------------------------------------- 
\subsection{Task 3.b: Routing Setup} 

The objective of this task is to configure the routing on Machines \texttt{A},
\texttt{B} and \texttt{R}, so \texttt{A} and \texttt{B} can communicate with 
each other. Students can use the following commands to 
set up the routing tables on these three machines. 


\begin{lstlisting}
// List all the entries in the routing table 
$ ip route 

// Add an entry 
$ sudo ip route add <network> dev <interface> via <router ip>

// Delete a routing entry
$ sudo ip route del ...
\end{lstlisting}


We need to configure \texttt{R} as a router.
Unless specifically configured, a computer will only act as a host,
not as a gateway. Machine \texttt{R} needs to forward packets, 
so it needs to function as a gateway. We need to
enable the IP forwarding for a computer to behave like a gateway.
IP forwarding can be enabled using the following command:

\begin{lstlisting}
$ sudo sysctl net.ipv4.ip_forward=1
\end{lstlisting}


In your lab report, please demonstrate (using screenshots) that you can successfully
\texttt{ping} and \texttt{telnet} to \texttt{B} from \texttt{A}, and 
vice versa. 
 


% -------------------------------------------
% SUBSECTION
% ------------------------------------------- 
\subsection{Task 3.c: Reverse Path Filtering}

Linux kernel implements a filtering rule called 
\textit{reverse path filtering},  which ensures the 
symmetric routing rule. When a packet with the source IP address X comes 
from an interface (say \texttt{I}), the OS will check whether the return
packet will return from the same interface, i.e., whether 
the routing for packets going to X is symmetric. 
To check that, the OS conducts a reverse lookup, 
finds out which interface will be used to route the return packets back to X.
If this interface is not \texttt{I}, i.e., different from where the 
original packet comes from, the routing path is asymmetric. In this case,
the kernel will drop the packet. 


The reverse path filtering mechanism can effectively
prevent outsiders from spoofing packets with the source IP 
address that belongs to the internal network. 
For example, in the Network depicted in 
Figure~\ref{ip:fig:internalnetwork}, Machine \texttt{A}
cannot send a spoofed packets to Machine \texttt{B} with a source IP 
address belonging to the \texttt{192.168.60.0/24} network.  


In this task, students will conduct an experiment to 
see the reverse path filtering in action. Students 
should send three spoofed packets on Machine \texttt{A}. 
All these packets should be sent to 
Machine \texttt{B}, but the source IP addresses should use one of 
the following: 


\begin{itemize}[noitemsep]
\item An IP address belonging to the network \texttt{10.0.2.0/24}.  
\item An IP address belonging to the internal network \texttt{192.168.60.0/24}.  
\item An IP address belonging to the Internet, such as \texttt{1.2.3.4}.
\end{itemize}


During the experiments, students run \texttt{Wireshark} on both Machine \texttt{B}
and \texttt{R}, and check whether the spoofed packets are forwarded to the internal network 
or not by \texttt{R}.
When running \texttt{Wireshark} on machine \texttt{R},
check the traffic on both network interfaces. If the spoofed packet
is forwarded, it should appear in both interfaces; otherwise, we will only see 
the packet from the incoming interface. Please describe and explain your observations. 




% *******************************************
% SECTION
% ******************************************* 
\section{Submission}

\seedsubmission

\end{document}



