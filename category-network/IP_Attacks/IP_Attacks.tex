%%%%%%%%%%%%%%%%%%%%%%%%%%%%%%%%%%%%%%%%%%%%%%%%%%%%%%%%%%%%%%%%%%%%%%
%%  Copyright by Wenliang Du.                                       %%
%%  This work is licensed under the Creative Commons                %%
%%  Attribution-NonCommercial-ShareAlike 4.0 International License. %%
%%  To view a copy of this license, visit                           %%
%%  http://creativecommons.org/licenses/by-nc-sa/4.0/.              %%
%%%%%%%%%%%%%%%%%%%%%%%%%%%%%%%%%%%%%%%%%%%%%%%%%%%%%%%%%%%%%%%%%%%%%%

\newcommand{\commonfolder}{../../common-files}
\documentclass[11pt]{article}

\usepackage[most]{tcolorbox}
\usepackage{times}
\usepackage{epsf}
\usepackage{epsfig}
\usepackage{amsmath, alltt, amssymb, xspace}
\usepackage{wrapfig}
\usepackage{fancyhdr}
\usepackage{url}
\usepackage{verbatim}
\usepackage{fancyvrb}
\usepackage{adjustbox}
\usepackage{listings}
\usepackage{color}
\usepackage{subfigure}
\usepackage{cite}
\usepackage{sidecap}
\usepackage{pifont}
\usepackage{mdframed}
\usepackage{textcomp}
\usepackage{enumitem}


% Horizontal alignment
\topmargin      -0.50in  % distance to headers
\oddsidemargin  0.0in
\evensidemargin 0.0in
\textwidth      6.5in
\textheight     8.9in 

\newcommand{\todo}[1]{
\vspace{0.1in}
\fbox{\parbox{6in}{TODO: #1}}
\vspace{0.1in}
}


\newcommand{\unix}{{\tt Unix}\xspace}
\newcommand{\linux}{{\tt Linux}\xspace}
\newcommand{\minix}{{\tt Minix}\xspace}
\newcommand{\ubuntu}{{\tt Ubuntu}\xspace}
\newcommand{\setuid}{{\tt Set-UID}\xspace}
\newcommand{\openssl} {\texttt{openssl}}


\pagestyle{fancy}
\lhead{\bfseries SEED Labs}
\chead{}
\rhead{\small \thepage}
\lfoot{}
\cfoot{}
\rfoot{}


\definecolor{dkgreen}{rgb}{0,0.6,0}
\definecolor{gray}{rgb}{0.5,0.5,0.5}
\definecolor{mauve}{rgb}{0.58,0,0.82}
\definecolor{lightgray}{gray}{0.90}


\lstset{%
  frame=none,
  language=,
  backgroundcolor=\color{lightgray},
  aboveskip=3mm,
  belowskip=3mm,
  showstringspaces=false,
%  columns=flexible,
  basicstyle={\small\ttfamily},
  numbers=none,
  numberstyle=\tiny\color{gray},
  keywordstyle=\color{blue},
  commentstyle=\color{dkgreen},
  stringstyle=\color{mauve},
  breaklines=true,
  breakatwhitespace=true,
  tabsize=3,
  columns=fullflexible,
  keepspaces=true,
  escapeinside={(*@}{@*)}
}

\newcommand{\newnote}[1]{
\vspace{0.1in}
\noindent
\fbox{\parbox{1.0\textwidth}{\textbf{Note:} #1}}
%\vspace{0.1in}
}


%% Submission
\newcommand{\seedsubmission}{You need to submit a detailed lab report, with screenshots,
to describe what you have done and what you have observed.
You also need to provide explanation
to the observations that are interesting or surprising.
Please also list the important code snippets followed by
explanation. Simply attaching code without any explanation will not
receive credits.}

%% Book
\newcommand{\seedbook}{\textit{Computer \& Internet Security: A Hands-on Approach}, 2nd
Edition, by Wenliang Du. See details at \url{https://www.handsonsecurity.net}.}

%% Videos
\newcommand{\seedisvideo}{\textit{Internet Security: A Hands-on Approach},
by Wenliang Du. See details at \url{https://www.handsonsecurity.net/video.html}.}

\newcommand{\seedcsvideo}{\textit{Computer Security: A Hands-on Approach},
by Wenliang Du. See details at \url{https://www.handsonsecurity.net/video.html}.}

%% Lab Environment
\newcommand{\seedenvironment}{This lab has been tested on our pre-built
Ubuntu 16.04 VM, which can be downloaded from the SEED website.}






\newcommand{\seedlabcopyright}[1]{
\vspace{0.1in}
\fbox{\parbox{6in}{\small Copyright \copyright\ {#1}\ \ by Wenliang Du.\\
      This work is licensed under a Creative Commons
      Attribution-NonCommercial-ShareAlike 4.0 International License.
      If you remix, transform, or build upon the material, 
      this copyright notice must be left intact, or reproduced in a way that is reasonable to
      the medium in which the work is being re-published.}}
\vspace{0.1in}
}





% Many labs reference the section numbers of the manual. If those numbers
% are changed, we need to change this file, otherwise, the reference will
% be out of sync.

\newcommand{\manualdocker}{1}
\newcommand{\manualonelan}{2}
\newcommand{\manualtwolans}{3}
\newcommand{\manualdns}{4}
\newcommand{\manualnaming}{5}
\newcommand{\manualproblem}{6}



\lhead{\bfseries SEED Labs -- IP/ICMP Attacks Lab}
\newcommand{\ipFigs}{./Figs}


\begin{document}



\begin{center}
{\LARGE IP/ICMP Attacks Lab}
\end{center}

\seedlabcopyright{2020}


% *******************************************
% SECTION
% ******************************************* 
\section{Overview}

The objective of this lab is for students to gain the first-hand experience 
on various attacks at the IP layer. Some of the attacks may not work anymore,
but their underlying techniques are quite generic, and it
is important for students to learn these attacking techniques, so when they design
or analyze network protocols, they are aware of what attackers can do to protocols.
Moreover, due to the complexity of IP fragmentation, spoofing fragmented IP packets
is non-trivial.  Constructing spoofed 
IP fragments is a good practice for students to hone their
packet spoofing skills, which are essential in network security.  
We will use Scapy to conduct packet spoofing.  
This lab covers the following topics:

\begin{itemize}[noitemsep]
\item The IP and ICMP protocols
\item IP Fragmentation and the related attacks
\item ICMP redirect attack
\item Routing 
\end{itemize}



\paragraph{Videos.}
Detailed coverage of the IP protocol and the attacks at the IP layer can be found 
in the following:

\begin{itemize}
\item Section 4 of the SEED Lecture, \seedisvideo
\end{itemize}


\paragraph{Lab environment.} \seedenvironmentB



% *******************************************
% SECTION
% *******************************************
\section{Environment Setup using Container}

In this lab, we need three machines. We use
containers to set up the lab environment, which is depicted
in Figure~\ref{ip:fig:labsetup}.
In this setup, we have an attacker machine (Host M),
which is used to launch attacks against the other two machines, Host A and
Host B.  These three machines must be on the same LAN,
because the ARP cache poisoning attack is limited to LAN.

\begin{figure}[htb]
\begin{center}
\includegraphics[width=0.8\textwidth]{\commonfolder/Figs/TwoLANs.pdf}
\end{center}
\caption{Lab environment setup}
\label{ip:fig:labsetup}
\end{figure}



% -------------------------------------------
% SUBSECTION
% -------------------------------------------
\subsection{Container Setup and Commands}

%%%%%%%%%%%%%%%%%%%%%%%%%%%%%%%%%%%%%%%%%%%%
Please download the
\texttt{Labsetup.zip} file to your VM from the lab's website,
unzip it, enter the \texttt{Labsetup} folder, and 
use the \texttt{docker-compose.yml} file to 
set up the lab environment. Detailed explanation
of the content in this file and all the involved 
\texttt{Dockerfile} can be found from the 
user manual, which is linked to the website of this lab.
If this is the first time you set up a SEED lab environment
using containers, it is very important that you read 
the user manual. 

In the following, we list some of the commonly
used commands related to Docker and Compose. 
Since we are going to use 
these commands very frequently, we have created aliases for them
in the \texttt{.bashrc} file (in our provided SEEDUbuntu 20.04 VM).


\begin{lstlisting}
$ docker-compose build  # Build the container image
$ docker-compose up     # Start the container
$ docker-compose down   # Shut down the container

// Aliases for the Compose commands above
$ dcbuild       # Alias for: docker-compose build
$ dcup          # Alias for: docker-compose up
$ dcdown        # Alias for: docker-compose down
\end{lstlisting}


All the containers will be running in the background. To run
commands on a container, we often need to get a shell on
that container. We first need to use the \texttt{"docker ps"}  
command to find out the ID of the container, and then
use \texttt{"docker exec"} to start a shell on that 
container. We have created aliases for them in
the \texttt{.bashrc} file.

\begin{lstlisting}
$ dockps        # Alias for: docker ps --format "{{.ID}}  {{.Names}}" 
$ docksh <id>   # Alias for: docker exec -it <id> /bin/bash

# The following example shows how to get a shell inside hostC
$ dockps
b1004832e275  hostA-10.9.0.5
0af4ea7a3e2e  hostB-10.9.0.6
9652715c8e0a  hostC-10.9.0.7

$ docksh 96
root@9652715c8e0a:/#  

# Note: If a docker command requires a container ID, you do not need to 
#       type the entire ID string. Typing the first few characters will 
#       be sufficient, as long as they are unique among all the containers. 
\end{lstlisting}


If you encounter problems when setting up the lab environment, 
please read the ``Common Problems'' section of the manual
for potential solutions.


%%%%%%%%%%%%%%%%%%%%%%%%%%%%%%%%%%%%%%%%%%%%


% -------------------------------------------
% SUBSECTION
% -------------------------------------------
\subsection{About the Attacker Container}

In this lab, we can either use the VM or the attacker container
as the attacker machine. If you look at the Docker Compose file, you will
see that the attacker container is configured differently from the other
containers. Here are the differences:

\begin{itemize}
\item \textit{Shared folder.} When we use the attacker container
to launch attacks, we need to put the attacking code inside
the attacker container.
%%%%%%%%%%%%%%%%%%%%%%%%%%%%%%%%%%%%%%%%%%%%%%%

\paragraph{Volumes.} In this lab, we need to write code and then run
the code inside containers. Code editing is more convenient inside
the VM, so we will do it from the VM. Using the Docker \texttt{volumes},
we can create a shared folder between VM and the container.
If you look at the \texttt{docker-compose.yml} file, you will find out that
we have added the following entry to the VPN client and server containers.
It indicates mounting the \texttt{./volumes} folder on the host
machine (i.e., the VM) to the \texttt{/volumes} folder inside the container.
We will write our code in the \texttt{./volumes} folder (on the VM), so they
can be used inside the containers.

\begin{lstlisting}
volumes:
       - ./volumes:/volumes
\end{lstlisting}


%%%%%%%%%%%%%%%%%%%%%%%%%%%%%%%%%%%%%%%%%%%%%%%

\item \textit{Privileged mode.}
%%%%%%%%%%%%%%%%%%%%%%%%%%%%%%%%%%%%%%%%%%%%%%%
To be able to modify kernel parameters at runtime (using \texttt{sysctl}),
such as enabling IP forwarding, a container needs to be privileged.
This is achieved by including the following entry
in the Docker Compose file for the container.

\begin{lstlisting}
privileged: true
\end{lstlisting}


%%%%%%%%%%%%%%%%%%%%%%%%%%%%%%%%%%%%%%%%%%%%%%%

\end{itemize}



% *******************************************
% SECTION
% ******************************************* 
\section{Task 1: IP Fragmentation}

We will only use two containers for this task, the attacker container
and the victim container.

% -------------------------------------------
% SUBSECTION
% ------------------------------------------- 
\subsection{Task 1.a: Conducting IP Fragmentation}

In this task, students need to construct a UDP packet and send it to a UDP 
server. They can use \texttt{"nc -lu -p 9090"} to start a UDP server inside
the victim container. 
Instead of building one single IP packet, students need to 
divide the packet into 3 fragments, each containing 32 bytes of data (the
first fragment contains 8 bytes of the UDP header plus 32 bytes of data).
If everything is done correctly, the server will display
96 bytes of data in total.  
The following is a sample code for constructing the first fragment.

\begin{lstlisting}
#!/usr/bin/env python3
from scapy.all import *

# Construct IP header
ip = IP(src="1.2.3.4", dst="10.9.0.5")
ip.id    = 1000  # Identification
ip.frag  = 0     # Offset of this IP fragment
ip.flags = 1     # Flags
ip.proto = 17    # For UDP

# Construct UDP header
udp = UDP(sport=7070, dport=9090)
udp.len  = 200   # This should be the combined length of all fragments

# Construct payload
payload = 'A' * 80    # Put 80 bytes in the first fragment

# Construct the entire packet and send it out
pkt = ip/udp/payload  # For other fragments, we should use ip/payload
pkt[UDP].chksum = 0 # Set the checksum field to zero
send(pkt, verbose=0)
\end{lstlisting}


It should be noted that the UDP checksum field needs to be set 
correctly. If we do not set this field, Scapy will calculate 
the checksum for us, but this checksum will only be based on the 
data in the first fragment, which is incorrect.
If we set the checksum field to  zero, Scapy will leave it alone.
Moreover, the recipient will not validate the UDP checksum 
if it sees a zero in the checksum field, 
because in UDP, checksum validation is optional.


If you use Wireshark to observe traffic, it should also be noted that by default, 
Wireshark will reassemble fragments in the last fragment packet and
show it as a complete IP/UDP packet. To change that behavior,
we should disable IP fragment reassembly in Wireshark preferences.
Click the following menu sequence: \texttt{Edit} $\rightarrow$ \texttt{Preferences}; 
click the \texttt{Protocols} dropdown menu, find and click \texttt{IPv4}.
Uncheck the \texttt{"Reassemble fragmented IPv4 datagrams"} option. 




% -------------------------------------------
% SUBSECTION
% ------------------------------------------- 
\subsection{Task 1.b: IP Fragments with Overlapping Contents}

Similar to Task 1.a, students also need to construct 3 fragments to send data to a UDP server.
The size of each fragment is up to students.  The objective of this task is to create
overlapping fragments.  In particular, the first two fragments should overlap.  Please use
experiments to show what will happen when the overlapping occurs. Please
try the following overlapping scenarios separately:
 
 \begin{itemize} 
 \item The end of the first fragment and the beginning of the second
 fragment should have \texttt{K} bytes of overlapping, i.e., the last  
 \texttt{K} bytes of data in the first fragment should have the same
 offsets as the first \texttt{K} bytes of data in the second fragment. 
 The value of \texttt{K} is decided by students (\texttt{K} should be 
 greater than zero and smaller than the size of either fragment). In the reports, students
 should indicate what their \texttt{K} values are. 


 \item The second fragment is completely enclosed in the first fragment.
 The size of the second fragment must be smaller than the 
 first fragment (they cannot be equal).

 \end{itemize} 


Please try two different orders: (1) sending the first fragment first, and 
(2) sending the second fragment first. Please report whether the results will
be the same. 





% -------------------------------------------
% SUBSECTION
% ------------------------------------------- 
\subsection{Task 1.c: Sending a Super-Large Packet}

As we know, the maximal size for an IP packet is $2^{16}$ octets, because
the length field in the IP header has only 16 bits. 
However,
using the IP fragmentation, we can create an IP packet that 
exceeds this limit. Please construct such a packet, send
it to the UDP server, and see how the server responds to this 
situation. Please report your observation. 
  



% -------------------------------------------
% SUBSECTION
% ------------------------------------------- 
\subsection{Task 1.d: Sending Incomplete IP Packet}


In this task, we are going to use Machine A to launch a Denial-of-Service attack 
on Machine B. In the attack, Machine A sends  a lot of 
incomplete IP packets to B, i.e., these packets consist of 
IP fragments, but some fragments are missing. All these incomplete IP packets 
will stay in the kernel, until they time out. Potentially, this can
cause the kernel to commit a lot of kernel memory. In the past, this 
had resulted in denial-of-service attacks on the server. Please try 
this attack and describe your observation. 




% *******************************************
% SECTION
% ******************************************* 
\section{Task 2: ICMP Redirect Attack}

An ICMP redirect is an error message sent by a router to the sender of an
IP packet. Redirects are used when a router believes a packet is being
routed incorrectly, and it would like to inform the sender that it should
use a different router for the subsequent packets sent to that same
destination.


In the Ubuntu operating system, 
there is a countermeasure against the ICMP redirect attack. In the 
Compose file, we have already 
turned off the countermeasure by configuring the victim container
to accept ICMP redirect messages. 

\begin{lstlisting}
// In docker-compose.yml
sysctls:
      - net.ipv4.conf.all.accept_redirects=1

// To turn the protection on, set its value to 0
# sysctl net.ipv4.conf.all.accept_redirects=0
\end{lstlisting}


For this task, we will attack the victim container from 
the attacker container. In the current setup, 
the victim will use the router container (\texttt{192.168.60.11}) as
the router to get to the \texttt{192.168.60.0/24} network. If
we run \texttt{ip route} on the victim container, we will
see the following

\begin{lstlisting}
# ip route
default via 10.9.0.1 dev eth0 
10.9.0.0/24 dev eth0 proto kernel scope link src 10.9.0.5 
192.168.60.0/24 via (*@\textbf{10.9.0.11}@*) dev eth0
\end{lstlisting}
 
The objective of this task is to launch an ICMP redirect attack on the victim,
such that when the victim sends packets to the \texttt{192.168.60.5}
network, it will use the malicious router container (\texttt{10.9.0.111})
as its router. 
Since the malicious router is controlled by the attacker, the attacker can 
intercept the packets, make changes, and then send 
the modified packets out. This is a form of the Man-In-The-Middle (MITM) attack. 
For the simplicity of this lab, students are not required to 
conduct the MITM part; they only need to demonstrate that 
their ICMP redirect attacks can successfully redirect
packets from A to B.


\paragraph{Code skeleton.} A code skeleton is provided in the following, with
some of the essential parameters left out. Students should fill in the proper 
values in the places marked by \texttt{@@@@}.  


\begin{lstlisting}
#!/usr/bin/python3

from scapy.all import *

ip = IP(src = @@@@,  dst = @@@@)
icmp = ICMP(type=@@@@, code=@@@@)
icmp.gw = @@@@

# The enclosed IP packet should be the one that 
# triggers the redirect message. 
ip2 = IP(src = @@@@, dst = @@@@)
send(ip/icmp/ip2/ICMP());
\end{lstlisting}
 

\paragraph{Verification.}
ICMP redirect messages will not affect the routing table; instead, it 
affects the routing cache. Entries in the routing cache overwrite 
those in the routing table, until the entries expire. To display 
and clean the cache contents, we can use the following commands: 

\begin{lstlisting}
// Display the routing cache 
# ip route show cache
192.168.60.5 via 10.9.0.111 dev eth0
    cache <redirected> expires 296sec

// Clean the routing cache
# ip route flush cache
\end{lstlisting}


Please do a traceroute on the victim machine, and see whether the packet
is rerouted or not. 

\begin{lstlisting}
# mtr -n 192.168.60.5
\end{lstlisting}
 



\paragraph{A strange issue.} While developing this lab, we have observed
a strange issue in the container environment. The issue does not exist
if the victim is a VM, instead of a container. 
If we spoof the redirect packets, but the victim machine is not 
sending out ICMP packets during the attack, the attack will never be successful. 
This is not the case for the VM setting. 
Moreover, the \texttt{ip2} inside the redirect packet must match with the
type and the destination IP address of the packets 
that the victim is currently sending (ICMP for ICMP,
UDP for UDP, etc.). 


It seems that the OS kernel conducts some kind of 
sanity check before accepting an ICMP redirect packets. 
We have not figured out what exactly caused this, 
and why the VM does not have these restrictions. 
This is an open issue for the SEED labs, and we encourage 
students to help us resolve this issue. We recommend instructors
to give students bonus points if they have indeed resolved this issue. 


Before we find a way to disable this checking mechanism, 
when we launch the attack,
we should should \texttt{ping} the \texttt{192.168.60.5} host on the 
victim machine. 


\paragraph{Questions.} After you have succeeded in the attack, please 
conduct the following experiments, and see whether your attack can 
still succeed. Please explain your observations:

\begin{enumerate}
\item Can you use ICMP redirect attacks to redirect to a remote machine? Namely,
the IP address assigned to \texttt{icmp.gw} is a computer not on the local LAN. 
Please show your experiment result, and explain your observation.  

\item Can you use ICMP redirect attacks to redirect to a non-existing machine on
the same network? Namely, the IP address assigned to \texttt{icmp.gw} is a local computer that
is either offline or non-existing. 
Please show your experiment result, and explain your observation.  

\item If you look at the \texttt{docker-compose.yml} file, you will find the 
following entries for the malicious router container. What are the purposes
of these entries? Please change their value to \texttt{1}, and launch the attack again. 
Please describe and explain your observation. 

\begin{lstlisting}
sysctls:
     - net.ipv4.conf.all.send_redirects=0
     - net.ipv4.conf.default.send_redirects=0
     - net.ipv4.conf.eth0.send_redirects=0
\end{lstlisting}
 
\end{enumerate}






% *******************************************
% SECTION
% ******************************************* 
\section{Submission}

%%%%%%%%%%%%%%%%%%%%%%%%%%%%%%%%%%%%%%%%

You need to submit a detailed lab report, with screenshots,
to describe what you have done and what you have observed.
You also need to provide explanation
to the observations that are interesting or surprising.
Please also list the important code snippets followed by
explanation. Simply attaching code without any explanation will not
receive credits.

%%%%%%%%%%%%%%%%%%%%%%%%%%%%%%%%%%%%%%%%


\end{document}



