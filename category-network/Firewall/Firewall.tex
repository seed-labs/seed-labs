%%%%%%%%%%%%%%%%%%%%%%%%%%%%%%%%%%%%%%%%%%%%%%%%%%%%%%%%%%%%%%%%%%%%%%
%%  Copyright by Wenliang Du.                                       %%
%%  This work is licensed under the Creative Commons                %%
%%  Attribution-NonCommercial-ShareAlike 4.0 International License. %%
%%  To view a copy of this license, visit                           %%
%%  http://creativecommons.org/licenses/by-nc-sa/4.0/.              %%
%%%%%%%%%%%%%%%%%%%%%%%%%%%%%%%%%%%%%%%%%%%%%%%%%%%%%%%%%%%%%%%%%%%%%%

\newcommand{\commonfolder}{../../common-files}
\documentclass[11pt]{article}

\usepackage[most]{tcolorbox}
\usepackage{times}
\usepackage{epsf}
\usepackage{epsfig}
\usepackage{amsmath, alltt, amssymb, xspace}
\usepackage{wrapfig}
\usepackage{fancyhdr}
\usepackage{url}
\usepackage{verbatim}
\usepackage{fancyvrb}
\usepackage{adjustbox}
\usepackage{listings}
\usepackage{color}
\usepackage{subfigure}
\usepackage{cite}
\usepackage{sidecap}
\usepackage{pifont}
\usepackage{mdframed}
\usepackage{textcomp}
\usepackage{enumitem}


% Horizontal alignment
\topmargin      -0.50in  % distance to headers
\oddsidemargin  0.0in
\evensidemargin 0.0in
\textwidth      6.5in
\textheight     8.9in 

\newcommand{\todo}[1]{
\vspace{0.1in}
\fbox{\parbox{6in}{TODO: #1}}
\vspace{0.1in}
}


\newcommand{\unix}{{\tt Unix}\xspace}
\newcommand{\linux}{{\tt Linux}\xspace}
\newcommand{\minix}{{\tt Minix}\xspace}
\newcommand{\ubuntu}{{\tt Ubuntu}\xspace}
\newcommand{\setuid}{{\tt Set-UID}\xspace}
\newcommand{\openssl} {\texttt{openssl}}


\pagestyle{fancy}
\lhead{\bfseries SEED Labs}
\chead{}
\rhead{\small \thepage}
\lfoot{}
\cfoot{}
\rfoot{}


\definecolor{dkgreen}{rgb}{0,0.6,0}
\definecolor{gray}{rgb}{0.5,0.5,0.5}
\definecolor{mauve}{rgb}{0.58,0,0.82}
\definecolor{lightgray}{gray}{0.90}


\lstset{%
  frame=none,
  language=,
  backgroundcolor=\color{lightgray},
  aboveskip=3mm,
  belowskip=3mm,
  showstringspaces=false,
%  columns=flexible,
  basicstyle={\small\ttfamily},
  numbers=none,
  numberstyle=\tiny\color{gray},
  keywordstyle=\color{blue},
  commentstyle=\color{dkgreen},
  stringstyle=\color{mauve},
  breaklines=true,
  breakatwhitespace=true,
  tabsize=3,
  columns=fullflexible,
  keepspaces=true,
  escapeinside={(*@}{@*)}
}

\newcommand{\newnote}[1]{
\vspace{0.1in}
\noindent
\fbox{\parbox{1.0\textwidth}{\textbf{Note:} #1}}
%\vspace{0.1in}
}


%% Submission
\newcommand{\seedsubmission}{You need to submit a detailed lab report, with screenshots,
to describe what you have done and what you have observed.
You also need to provide explanation
to the observations that are interesting or surprising.
Please also list the important code snippets followed by
explanation. Simply attaching code without any explanation will not
receive credits.}

%% Book
\newcommand{\seedbook}{\textit{Computer \& Internet Security: A Hands-on Approach}, 2nd
Edition, by Wenliang Du. See details at \url{https://www.handsonsecurity.net}.}

%% Videos
\newcommand{\seedisvideo}{\textit{Internet Security: A Hands-on Approach},
by Wenliang Du. See details at \url{https://www.handsonsecurity.net/video.html}.}

\newcommand{\seedcsvideo}{\textit{Computer Security: A Hands-on Approach},
by Wenliang Du. See details at \url{https://www.handsonsecurity.net/video.html}.}

%% Lab Environment
\newcommand{\seedenvironment}{This lab has been tested on our pre-built
Ubuntu 16.04 VM, which can be downloaded from the SEED website.}






\newcommand{\seedlabcopyright}[1]{
\vspace{0.1in}
\fbox{\parbox{6in}{\small Copyright \copyright\ {#1}\ \ by Wenliang Du.\\
      This work is licensed under a Creative Commons
      Attribution-NonCommercial-ShareAlike 4.0 International License.
      If you remix, transform, or build upon the material, 
      this copyright notice must be left intact, or reproduced in a way that is reasonable to
      the medium in which the work is being re-published.}}
\vspace{0.1in}
}






\newcommand{\telnet} {\texttt{telnet}\xspace}
\newcommand{\iptables}{\texttt{iptables}\xspace}
\newcommand{\netfilter}{\texttt{netfilter}\xspace}
\newcommand{\Netfilter}{\texttt{Netfilter}\xspace}

\newcommand{\firewallFigs}{./Figs}
\lhead{\bfseries SEED Labs -- Firewall Lab}

\newcommand{\pointleft}[1]{\reflectbox{\ding{217}} \textbf{\texttt{#1}}}

\begin{document}



\begin{center}
{\LARGE Firewall Lab}
\end{center}

\seedlabcopyright{2006 - 2020}



% *******************************************
% SECTION
% ******************************************* 
\section{Overview}

The learning objective of this lab is two-fold: learning
how firewalls work, and setting up a simple firewall
for a network. Students will first 
implement a simple stateless packet-filtering firewall, 
which inspects packets, and decides 
whether to drop or forward a packet based on firewall rules. 
Through this implementation task, students can get the 
basic ideas on how firewall works.


Actually, Linux already has a built-in firewall, also based on 
\texttt{netfilter}. This firewall is called \iptables. 
Students will be given a simple network topology, and are asked to
use \iptables to set up firewall rules to protect the network. 
Students will also be exposed to several other interesting 
applications of \iptables. 
This lab covers the following topics:


\begin{itemize}[noitemsep]
\item Firewall
\item Netfilter
\item Loadable kernel module
\item Using \iptables to set up firewall rules
\item Various applications of \iptables
\end{itemize}


\paragraph{Readings and videos.}
Detailed coverage of firewalls can be found in the following:

\begin{itemize}
\item Chapter 17 of the SEED Book, \seedbook
\item Section 9 of the SEED Lecture, \seedisvideo
\end{itemize}


\paragraph{Lab environment.} \seedenvironmentB




% *******************************************
% SECTION
% ******************************************* 
\section{Environment Setup Using Containers}


In this lab, we need to use multiple machines. 
Their setup is depicted in Figure~\ref{fig:labsetup}.  
We will use containers to set up this lab environment.


\begin{figure}[htb]
\begin{center}
\includegraphics[width=0.8\textwidth]{./Figs/TwoLANs.pdf}
\end{center}
\caption{Lab setup}
\label{fig:labsetup}
\end{figure}


% -------------------------------------------
% SUBSECTION
% -------------------------------------------
\subsection{Container Setup and Commands}
%%%%%%%%%%%%%%%%%%%%%%%%%%%%%%%%%%%%%%%%%%%%
Please download the
\texttt{Labsetup.zip} file to your VM from the lab's website,
unzip it, enter the \texttt{Labsetup} folder, and 
use the \texttt{docker-compose.yml} file to 
set up the lab environment. Detailed explanation
of the content in this file and all the involved 
\texttt{Dockerfile} can be found from the 
user manual, which is linked to the website of this lab.
If this is the first time you set up a SEED lab environment
using containers, it is very important that you read 
the user manual. 

In the following, we list some of the commonly
used commands related to Docker and Compose. 
Since we are going to use 
these commands very frequently, we have created aliases for them
in the \texttt{.bashrc} file (in our provided SEEDUbuntu 20.04 VM).


\begin{lstlisting}
$ docker-compose build  # Build the container image
$ docker-compose up     # Start the container
$ docker-compose down   # Shut down the container

// Aliases for the Compose commands above
$ dcbuild       # Alias for: docker-compose build
$ dcup          # Alias for: docker-compose up
$ dcdown        # Alias for: docker-compose down
\end{lstlisting}


All the containers will be running in the background. To run
commands on a container, we often need to get a shell on
that container. We first need to use the \texttt{"docker ps"}  
command to find out the ID of the container, and then
use \texttt{"docker exec"} to start a shell on that 
container. We have created aliases for them in
the \texttt{.bashrc} file.

\begin{lstlisting}
$ dockps        # Alias for: docker ps --format "{{.ID}}  {{.Names}}" 
$ docksh <id>   # Alias for: docker exec -it <id> /bin/bash

# The following example shows how to get a shell inside hostC
$ dockps
b1004832e275  hostA-10.9.0.5
0af4ea7a3e2e  hostB-10.9.0.6
9652715c8e0a  hostC-10.9.0.7

$ docksh 96
root@9652715c8e0a:/#  

# Note: If a docker command requires a container ID, you do not need to 
#       type the entire ID string. Typing the first few characters will 
#       be sufficient, as long as they are unique among all the containers. 
\end{lstlisting}


If you encounter problems when setting up the lab environment, 
please read the ``Common Problems'' section of the manual
for potential solutions.


%%%%%%%%%%%%%%%%%%%%%%%%%%%%%%%%%%%%%%%%%%%%



% -------------------------------------------
% SUBSECTION
% -------------------------------------------
\subsection{Detach the VM from \texttt{192.168.60.0/24} Network} 
%%%%%%%%%%%%%%%%%%%%%%%%%%%%%%%%%%%%%%%%%%%%

%\paragraph{Removing the Host VM from \texttt{192.168.60.0/24} network.}
When Docker creates a network, it automatically attach the
host machine (i.e., the VM) to the network. In this
case, the host machine is attached to both
networks, and is given the IP address \texttt{10.9.0.1} and
\texttt{192.168.60.1}.
This may create undesirable situations,
so let's remove it.

In our setup, we only want the VM to be connected to
\texttt{10.9.0.0/24}, not to both.
We need to detach the VM from the other network. We cannot simply power down
the interface, because that will affects all the containers on that network.
Our solution is to remove the IP address
of the interface, and then set the routing accordingly. 
script does exactly that.


\begin{lstlisting}
// Get the name of the interface 
$ ip addr

// Remove the IP address from the interface
$ sudo ip addr flush <interface-name>

// Use 10.9.0.11 as the router to access 192.168.60.0/24
$ sudo ip route add 192.168.60.0/24 via 10.9.0.11
\end{lstlisting}


%%%%%%%%%%%%%%%%%%%%%%%%%%%%%%%%%%%%%%%%%%%%



% *******************************************
% SECTION
% ******************************************* 
\section{Task 1: Implementing a Simple Firewall} 


In this task, we will implement a simple packet filtering 
type of firewall, which 
inspects each incoming and outgoing packets, and enforces the firewall policies 
set by the administrator. Since the packet 
processing is done within the kernel, the filtering must also be 
done within the kernel. Therefore, it seems that implementing such
a firewall requires us to modify the \linux kernel. In the past, 
this had to be done by modifying and rebuilding 
the kernel. The modern \linux 
operating systems provide several new mechanisms 
to facilitate the manipulation of packets without rebuilding
the kernel image. These two mechanisms are 
\textit{Loadable Kernel Module} (\texttt{LKM}) and \texttt{Netfilter}.



% -------------------------------------------
% SUBSECTION
% ------------------------------------------- 
\subsection{Task 1.a: Implement a Simple Kernel Module}


{\tt LKM} allows us to add a new module to the kernel at the runtime. 
This new module enables us to extend the functionalities of the kernel,
without rebuilding the kernel or even rebooting the computer. 
The packet filtering part of a firewall can be implemented as an LKM. 
In this task, we will get familiar with LKM.


The following is a simple loadable kernel module. It prints out 
\texttt{"Hello World!"} when the module is loaded; when the module
is removed from the kernel, it prints out \texttt{"Bye-bye World!"}.
The messages are not printed out on the screen; they are 
actually printed into the \texttt{/var/log/syslog} file. You can
use \texttt{"dmesg | tail -10"} to read the last 10 lines of 
the messages. 


\begin{lstlisting}
#include <linux/module.h>
#include <linux/kernel.h>

int init_module(void)
{
    printk(KERN_INFO "Hello World!\n");
    return 0;
}

void cleanup_module(void)
{
    printk(KERN_INFO "Bye-bye World!.\n");
}
\end{lstlisting}

We now need to create {\tt Makefile}, which includes the following
contents (the above program is named {\tt hello.c}). Then 
just type {\tt make}, and the above program will be compiled
into a loadable kernel module (when you copy and paste the following
into \texttt{Makefile}, make sure replace the spaces before the 
\texttt{make} commands with a tab).


\begin{lstlisting}
obj-m += hello.o

all:
        make -C /lib/modules/$(shell uname -r)/build M=$(PWD) modules

clean:
        make -C /lib/modules/$(shell uname -r)/build M=$(PWD) clean
\end{lstlisting}


Once the module is built by typing {\tt make}, you can use the following commands to 
load the module, list all modules, and remove the module. 
Also, you can use {\tt modinfo mymod.ko} to show information about a 
Linux Kernel module.

\begin{lstlisting}
# insmod mymod.ko        (inserting a module)
# lsmod                  (list all modules)
# rmmod mymod.ko         (remove the module)
# dmesg | tail -10       (check the messages)
\end{lstlisting}


\paragraph{Task.} Please compile this simple kernel module on 
your VM, and run it on the router container. 
Please show your running results in the lab report. 


% -------------------------------------------
% SUBSECTION
% ------------------------------------------- 
\subsection{Task 1.b: Implement a Simple Firewall Using \texttt{Netfilter}}  


In this task, we will write our packet filtering program
as an LKM, and then insert in into the packet processing path
inside the kernel. This cannot be easily done in the past before 
the \Netfilter was introduced into the \linux.

{\tt Netfilter} is designed to facilitate the manipulation of 
packets by authorized users. {\tt Netfilter} achieves this 
goal by implementing a number of {\em hooks} in the 
\linux kernel. These hooks are inserted into various places, 
including the packet incoming and outgoing paths. 
If we want to manipulate the incoming packets, we simply
need to connect our own programs (within LKM) to the 
corresponding hooks. Once an incoming packet arrives, 
our program will be invoked. Our program can decide 
whether this packet should be blocked or not; moreover,
we can also modify the packets in the program.


In this task, you need to use LKM and {\tt Netfilter} to implement
the packet filtering module.  This module will fetch 
the firewall policies from a data structure, and use the 
policies to decide whether packets should be blocked or not.
To make your life easier, so you can focus on the filtering part, 
the core of firewalls, we allow you to hardcode your firewall policies 
in the program. You should support at least three different 
rules. Guidelines on how to use \texttt{Netfilter} can be 
found in Chapter 17 of the SEED book.


Using {\tt Netfilter} is quite straightforward. All we need to do
is to hook our functions (in the kernel module) to the corresponding
{\tt Netfilter} hooks. Here we show an example (the code is in
\path{Labsetup/packet_filter}):


\begin{lstlisting}
#include <linux/module.h>
#include <linux/kernel.h>
#include <linux/netfilter.h>
#include <linux/netfilter_ipv4.h>
#include <linux/ip.h>
#include <linux/tcp.h>


/* This is the structure we shall use to register our function */
static struct nf_hook_ops nfho;

/* This is the hook function itself */
unsigned int hook_func(void *priv, struct sk_buff *skb, 
                       const struct nf_hook_state *state)

{
    /* This is where you can inspect the packet contained in
       the structure pointed by skb, and decide whether to accept 
       or drop it. You can even modify the packet */
 
    struct iphdr *iph;
    struct tcphdr *tcph;
    
    iph = ip_hdr(skb);
    tcph = (void *)iph+iph->ihl*4;
 
    if (iph->protocol == IPPROTO_TCP && tcph->dest == htons(23)) {
      printk(KERN_INFO "Dropping telnet packet to %d.%d.%d.%d\n",
          ((unsigned char *)&iph->daddr)[0],
          ((unsigned char *)&iph->daddr)[1],
          ((unsigned char *)&iph->daddr)[2],
          ((unsigned char *)&iph->daddr)[3]);
      return NF_DROP;       /* Drop this packet */
    } else {
      return NF_ACCEPT;     /* Accept this packet */
    }
}

/* Initialization routine */
int init_module()
{   /* Fill in our hook structure */
    nfho.hook = hook_func;               // Handler function 
    nfho.hooknum  = NF_INET_PRE_ROUTING; // The hook 
    nfho.pf       = PF_INET;
    nfho.priority = NF_IP_PRI_FIRST;     // Make our function first 

    // nf_register_hook(&nfho);               // For Ubuntu 16.04 VM
    nf_register_net_hook(&init_net, &nfho);   // For Ubuntu 20.04 VM
    return 0;
}

/* Cleanup routine */
void cleanup_module()
{
    // nf_unregister_hook(&nfho);             // For Ubuntu 16.04 VM
    nf_unregister_net_hook(&init_net, &nfho); // For Ubuntu 20.04 VM
}
\end{lstlisting}


\paragraph{Note for Ubuntu 20.04 VM:}
The code in the SEED book was developed in Ubuntu 16.04. It needs to be changed slightly 
to work in Ubuntu 20.04. The change is in the hook registration and 
unregistration APIs. See the difference in the following:


\begin{lstlisting}
// Hook registration:
  nf_register_hook(&nfho);                  // For Ubuntu 16.04 VM
  nf_register_net_hook(&init_net, &nfho);   // For Ubuntu 20.04 VM


// Hook unregistration:
  nf_unregister_hook(&nfho);                // For Ubuntu 16.04 VM
  nf_unregister_net_hook(&init_net, &nfho); // For Ubuntu 20.04 VM
\end{lstlisting}
 

\paragraph{Tasks.} Based on the code provided above (they can be downloaded
from the lab's website), please do the following tasks:

\begin{itemize}
\item Compile and run the provided code. Based on your execution results,
explain the purpose of the firewall.  

\item The following are \netfilter hooks. Please replace the 
hook used in the sample code with the ones marked by \ding{81}
and describe your observation.

\begin{lstlisting}
NF_INET_PRE_ROUTING 
NF_INET_LOCAL_IN         (*@\ding{81}@*)
NF_INET_FORWARD 
NF_INET_LOCAL_OUT 
NF_INET_POST_ROUTING     (*@\ding{81}@*)
\end{lstlisting}
\end{itemize}
 

\paragraph{Account:} 
seed and root account on container. 


% *******************************************
% SECTION
% ******************************************* 
\section{Task 2: Experimenting with Stateless Firewall Rules}

In the previous task, we had a chance to build a simple firewall using \netfilter. Actually,
\linux already has a built-in firewall, also based on \netfilter. This firewall is called
\iptables. Technically, the kernel part implementation of the firewall
is called \texttt{Xtables}, while \iptables is a user-space program to
configure the firewall. However, \iptables is often used to refer to both the kernel-part
implementation and the user-space program. 



% -------------------------------------------
% SUBSECTION
% ------------------------------------------- 
\subsection{Background of \iptables}

In this task, we will use \iptables to set up a firewall. 
The \iptables firewall is designed not only to filter packets, but also to make changes to
packets. To help manage these firewall rules for different purposes, \iptables organizes all
rules using a hierarchical structure: table, chain, and rules.
There are several tables, each specifying the main purpose of the rules as shown
in Table~\ref{firewall:table:iptables}.
For example, rules for packet filtering should be
placed in the \texttt{filter} table, while rules for making changes to packets should be placed
in the \texttt{nat} or \texttt{mangle} tables.

Each table contains several chains, each of which corresponds to a \netfilter hook. Basically,
each chain indicates where its rules are enforced. For example, rules on
the \texttt{FORWARD} chain are enforced at the \texttt{NF\_INET\_FORWARD} hook, and rules on
the \texttt{INPUT} chain are enforced at the  \texttt{NF\_INET\_LOCAL\_IN} hook.

Each chain contains a set of firewall rules that will be enforced.
When we set up firewalls, we add rules to these chains.
For example, if we would like to block all incoming \telnet traffic, we would
add a rule to the \texttt{INPUT} chain of the \texttt{filter} table.  If we
would like to redirect all incoming \telnet traffic to a different
port on a different host, basically doing port forwarding, we can add a rule to the
\texttt{INPUT} chain of the \texttt{mangle} table, as we need to make changes to packets.


\begin{table}[htb]
        \centering
%       \renewcommand{\arraystretch}{1.2}
        \caption{\iptables Tables and Chains}
        \label{firewall:table:iptables}
        \centering

        \begin{tabular}{|l|l|l|}
                \hline
                \bfseries Table & \bfseries Chain & \bfseries Functionality \\
                \hline\hline
                filter          &    \texttt{INPUT}      & Packet filtering \\
                                &    \texttt{FORWARD}    & \\
                                &    \texttt{OUTPUT}      & \\
                \hline
                nat             &   \texttt{PREROUTING}    & Modifying source or destination \\
                                &   \texttt{INPUT}      & network addresses \\
                                &   \texttt{OUTPUT}      & \\
                                &   \texttt{POSTROUTING}   & \\
                \hline
                mangle          &   \texttt{PREROUTING}    & Packet content modification \\
                                &   \texttt{INPUT}      & \\
                                &   \texttt{FORWARD}     & \\
                                &   \texttt{OUTPUT}      & \\
                                &   \texttt{POSTROUTING}   & \\
                \hline
        \end{tabular}
\end{table}


% -------------------------------------------
% SUBSECTION
% ------------------------------------------- 
\subsection{Using \iptables}


To add rules to the chains in each table, we use the \iptables command,
which is a quite powerful command. 
Students can find the manual of \iptables by typing \texttt{"man iptables"} 
or easily find many tutorials from online. 
What makes \iptables complicated is the many command-line arguments 
that we need to provide when
using the command. However, 
if we understand the structure of these command-line arguments, 
we will find out that the command is not that complicated. 


In a typical \iptables command, we add a rule to or remove a rule 
from one of the chains in one of the tables, so we need to 
specify a table name (the default is \texttt{filter}), a chain name, 
and an operation on the chain. After that, we specify the rule, which
is basically a pattern that will be matched with each of the 
packets passing through. If there is a match, an action will be 
performed on this packet. 
The general structure of the command is depicted in the following:

\begin{lstlisting}
iptables -t <table> -<operation> <chain>  <rule>   -j <target>
         ---------- --------------------  -------  -----------
              Table          Chain           Rule      Action
\end{lstlisting}


The rule is the most complicated part of the \iptables command. 
We will provide additional information later on when we use 
specific rules. In the following, we list some commonly 
used commands: 


\begin{lstlisting}
// List all the rules in a table (without line number)
iptables -t nat -L -n

// List all the rules in a table (with line number)
iptables -t filter -L -n --line-numbers

// Delete rule No. 2 in the INPUT chain of the filter table 
iptables -t filter -D INPUT 2

// Drop all the incoming packets that satisfy the <rule>
iptables -t filter -A INPUT <rule>  -j DROP
\end{lstlisting}


\paragraph{Note.} Docker relies on \iptables to manage 
the networks it creates, so it adds many rules to 
the \texttt{nat} table. 
When we manipulate \iptables rules, we should be careful 
not to remove Docker rules. For example, it will be quite
dangerous to run the \texttt{"iptables -t nat -F"} command, because 
it removes all the rules in the \texttt{nat} table,
including many of the Docker rules. That will cause 
trouble to Docker containers. Doing this for 
the \texttt{filter} table is fine, because Docker by default 
does not add any rule in this table. 


% -------------------------------------------
% SUBSECTION
% -------------------------------------------
\subsection{Task 2.A: Protecting the Router} 

In this task, we will set up rules to prevent outside machines from 
accessing the router machine, expect ping.   
Please execute the following \iptables
command on the router container, and then trying to 
access it from \texttt{10.9.0.5}. (1) Can you ping 
the router? (2) Can you telnet into the router (a 
telnet server is running on the router container). 
Please report your observation and explain the purpose for 
each rule. 

\begin{lstlisting}
iptables -A INPUT  -p icmp --icmp-type echo-reply   -j ACCEPT
iptables -A OUTPUT -p icmp --icmp-type echo-request -j ACCEPT
iptables -P OUTPUT DROP     (*@\pointleft{Set default rule for OUTPUT}@*)
iptables -P INPUT  DROP     (*@\pointleft{Set default rule for INPUT}@*)
\end{lstlisting}
 

\paragraph{Cleanup.} 
Before moving on to the next task, please restore the \texttt{filter} 
talbe to its original state by running the following commands:

\begin{lstlisting}
iptables -F
iptables -P OUTPUT ACCEPT
iptables -P INPUT  ACCEPT
\end{lstlisting}
 
Another way to restore the states of all the tables is to restart the 
containter. You can do it using the following command (you need 
to find the container's ID first):

\begin{lstlisting}
$ docker restart <Container ID>
\end{lstlisting}
 


% -------------------------------------------
% SUBSECTION
% -------------------------------------------
\subsection{Task 2.B: Protecting the Internal Network} 

In this task, we will set up firewall ruls on the router to protect the 
internal network \texttt{192.168.60.0/24}. We need to use the 
FORWARD table for this purpose. 

The directions of packets in the INPUT and OUTPUT tables are clear,
they are either coming into (for INPUT) or going out (for OUTPUT). 
This is not true for the FORWARD table, because it is 
bi-directional: packets going into
the internal network or going out to the external network
all go through the chains in this table. To specify the direction,
we can add the interface options using \texttt{"-i xyz"} (coming in from 
the \texttt{xyz} interface) and/or \texttt{"-o xyz"} (going out 
from the \texttt{xyz} interface).  


In this task, we want to implement a firewall to protect the 
internal network. More specifically, we need to enforce the 
the following restrictions on the ICMP traffic: 

\begin{enumerate}[noitemsep]
  \item Outside hosts cannot ping internal hosts. 
  \item Outside hosts can ping the router.
  \item Internal hosts can ping outside hosts. 
  \item All other packets from between the internal and external networks should be blocked.
\end{enumerate}

You will need to use the \texttt{"-p icmp"} options to specify the match
options related to the ICMP protocol. You can run 
\texttt{"iptables -p icmp -h"} to find out all the ICMP match
options. The following example drops the ICMP echo request.


\begin{lstlisting}
iptables -A FORWARD -p icmp --icmp-type echo-request -j DROP
\end{lstlisting}

When you are done with this task,
please remember to clean the table or restart the container 
before moving on to the next task.


% -------------------------------------------
% SUBSECTION
% -------------------------------------------
\subsection{Task 2.C: Protecting Internal Servers}

In this task, we want to protect the TCP servers 
inside the internal network (\texttt{192.168.60.0/24}). 
More specifically, we would like to achieve the following objectives.

\begin{enumerate}[noitemsep]
  \item Outside hosts can reach the web servers in the internal network.
    We assume that these servers are listening to port \texttt{8080}.
    There is no need to run an actual web server. You can use 
    \texttt{"nc -lk 8080"} to start a TCP server at port \texttt{8080}.  

  \item Outside hosts cannot reach the internal servers 
    that are listening to other ports.

  \item Internal hosts can reach all the internal servers.

  \item Internal hosts can also reach all the outside servers, except 
    the telnet server that listens to port \texttt{23}. In our setup,
    all the containers run a telnet server, so you can use them 
    to test your firewall.

  \item In this task, the connection tracking mechanism is not allowd. 
    It will bed use in a later task. 
\end{enumerate}

You will need to use the \texttt{"-p tcp"} options to specify the match
options related to the TCP protocol. You can run 
\texttt{"iptables -p tcp -h"} to find out all the TCP match
options. The following example allows the TCP packets coming from
the interface \texttt{eth0} if their source port is \texttt{5000}.  

\begin{lstlisting}
iptables -A FORWARD -i eth0 -p tcp --sport 5000  -j ACCEPT
\end{lstlisting}


When you are done with this task,
please remember to clean the table or restart the container 
before moving on to the next task.




% *******************************************
% SECTION
% ******************************************* 
\section{Task 3: Connection Tracking and Stateful Firewall}


In the previous task, we have only set up stateless firewalls, which inspect each
packet independently. However, packets
are usually not independent; they may be part of a TCP connection,
or they may be ICMP packets triggered by other packets. Treating them
independently does not take into consideration the context of the
packets, and can thus lead to inaccurate or unsafe firewall rules.
For example, if we would like to allow TCP packets to get into our network
only if a connection was made first, we cannot achieve that easily 
using stateless packet filters, because when the firewall examines each individual TCP packet,
it has no idea about whether the packet belongs to an existing connection
or not, unless the firewall maintains some state information for each connection.
If it does that, it becomes a stateful firewall.


% -------------------------------------------
% SUBSECTION
% ------------------------------------------- 
\subsection{Task 3.A: Experiment with the Connection Tracking} 


To support stateful firewalls, we need to be able to track connections. 
This is achieved by the \texttt{conntrack} mechanism inside the kernel. 
In this task, we will conduct experiments related to this module, and 
get familiar with the connection tracking mechanism. 
In our experiment, we will check the connection tracking information
on the router container. This can be done using the following command: 

\begin{lstlisting}
# conntrack -L
\end{lstlisting}

The goal of the task is to use a series of experiments to 
help students understand the 
connection concept in this tracking mechanism, especially
for the ICMP and UDP protocols, because unlike TCP,  they 
do not have connections. 


Please conduct the following experiments. For each experiment, please 
describe your observation, along with your explanation. 

\begin{itemize}
\item ICMP experiment: Run the following command and 
check the connection tracking information on the router. Describe
your observation. How long is the ICMP connection state be kept? 

\begin{lstlisting}
// On 10.9.0.5, send out ICMP packets
# ping 192.168.60.5
\end{lstlisting}

\item UDP experiment: Run the following command and 
check the connection tracking information on the router. Describe
your observation. How long is the UDP connection state be kept? 


\begin{lstlisting}
// On 192.168.60.5, start a netcat UDP server
# nc -lu 9090

// On 10.9.0.5, send out UDP packets 
# nc -u 192.168.60.5 9090
\end{lstlisting}


\item TCP experiment: Run the following command and 
check the connection tracking information on the router. Describe
your observation. How long is the TCP connection state be kept? 

\begin{lstlisting}
// On 192.168.60.5, start a netcat TCP server
# nc -l 9090

// On 10.9.0.5, send out TCP packets 
# nc 10.9.0.5 9090
\end{lstlisting}

\end{itemize}
 


% -------------------------------------------
% SUBSECTION
% ------------------------------------------- 
\subsection{Task 3.B: Setting Up a Stateful Firewall} 


Now we are ready to set up firewall rules based on connections. 
In the following example, 
the \texttt{"-m conntrack"} option indicates that we are using the \texttt{conntrack} module,
which is a very important module for \iptables; it tracks connections, and
\iptables replies on the tracking information to build stateful firewalls. 
The \texttt{--ctsate ESTABLISHED,RELATED} indicates that whether a packet
belongs to an \texttt{ESTABLISHED} or \texttt{RELATED} connection.
The rule allows TCP packets belonging to an existing connection to 
pass through. 

\begin{lstlisting}
iptables -A FORWARD -p tcp -m conntrack          \
         --ctstate ESTABLISHED,RELATED -j ACCEPT
\end{lstlisting}


Please improve the firewall rules from Task 2.C using the \texttt{conntrack} module. 
Please explain its advantage or the stateless firewall rules. Are there 
any disadvantages? Please explain.



% *******************************************
% SECTION
% ******************************************* 
\section{Task 4: Limiting Network Traffic}

In addition to blocking packets, we can also 
limit the number of packets that can pass through the firewall. 
This can be done using the \texttt{limit} module of \iptables.
In this task, we will use this module to limit how many packets 
from \texttt{10.9.0.5} are allowed to get into the internal network. 
You can use \texttt{"iptables -m limit -h"} to see the manual.  

\begin{lstlisting}
$ iptables -m limit -h
limit match options:
--limit avg             max average match rate: default 3/hour
                        [Packets per second unless followed by
                        /sec /minute /hour /day postfixes]
--limit-burst number    number to match in a burst, default 5
\end{lstlisting}
 

Please run the following commands on router, and then
ping \texttt{192.168.60.5} from \texttt{10.9.0.5}.  
Describe your observation. 
Please conduct the experiment with and without the second command, 
and then explain whether the second rule is needed or not, and why.

\begin{lstlisting}
iptables -A FORWARD -m limit   \
         --limit 10/minute --limit-burst 5 -j ACCEPT

iptables -A FORWARD -j DROP
\end{lstlisting}



% *******************************************
% SECTION
% *******************************************
\section{Task 5: Load Balancing}

The \iptables is very powerful. In addition to firewalls,
it has many other applications. We will not be able to 
cover all its applications in this lab, but we will experimenting
with one of the applications, load balancing. In this task,
we will use it to load balance three UDP servers running in the 
internal network. Let's first start the server
on each of the hosts: \texttt{192.168.60.5}, \texttt{192.168.60.6}, and 
\texttt{192.168.60.7}: 

\begin{lstlisting}
nc -lu 8080
\end{lstlisting}

We can use the \texttt{statistic} module to achieve load balancing. 
You can type the following command to get its manual. You can 
see there are two modes: \texttt{random} and \texttt{nth}. 
We will conduct experiments using both of them.

\begin{lstlisting}
$ iptables -m statistic -h 
statistic match options:
 --mode mode         Match mode (random, nth)
 random mode:
[!] --probability p  Probability
 nth mode:
[!] --every n        Match every nth packet
 --packet p          Initial counter value (0 <= p <= n-1, default 0)
\end{lstlisting}
 

\paragraph{Using the \texttt{nth} mode (round-rubin).}
On the router container, we set the following rule, which applies 
to all the UDP packets going to port \texttt{8080}. 
The \texttt{nth} mode of the \texttt{statistic} module is used; 
it implements a round-rubin load balancing policy: for every
three packets, pick the packet 0 (i.e., the first one), 
change its destination IP address and port number to 
\texttt{192.168.60.5}  and \texttt{8080}, respectively.  
The modified packets will continue on its journey.

\begin{lstlisting}
iptables -t nat -A PREROUTING -p udp --dport 8080      \
         -m statistic --mode nth --every 3 --packet 0  \
         -j DNAT --to-destination 192.168.60.5:8080
\end{lstlisting}

It should be noted that those packets that do not match
the rule will continue on their journeys; they will
not be modified or blocked. With this rule in place, 
if you send a UDP packet to the router's \texttt{8080} port,  
you will see that one out of three packets gets to 
\texttt{192.168.60.5}. 

\begin{lstlisting}
// On 10.9.0.5
echo hello | nc -u 10.9.0.11 8080
\end{lstlisting}
 

Please add more rules to the router container, 
so all the three internal hosts get the equal 
number of packets. 
Please provide some explanation for the rules. 


\paragraph{Using the \texttt{random} mode.}
Let's use a different mode to achieve the load balancing. The following 
rule will select a matching packet with the probability \texttt{P}.  
You need to replace \texttt{P} with a probability number.

\begin{lstlisting}
iptables -t nat -A PREROUTING -p udp --dport 8080   \
         -m statistic --mode random --probability P \
         -j DNAT --to-destination 192.168.60.5:8080
\end{lstlisting}

Please use this mode to implement your load balancing 
rules, so each internal server get roughly the 
same amount of traffic (it may not be exactly the same, 
but should be close when the total number of packets is large). 
Please provide some explanation for the rules. 


\paragraph{Further thoughts.}
Loading balancing on TCP is much more complicated, 
because we need to ensure that the packets in a TCP connections 
will go to the same server. This requires the help from 
the connection tracking. Students who are interested can
pursue this on their own. 


% *******************************************
% SECTION
% ******************************************* 
\section{Submission and Demonstration}


%%%%%%%%%%%%%%%%%%%%%%%%%%%%%%%%%%%%%%%%

You need to submit a detailed lab report, with screenshots,
to describe what you have done and what you have observed.
You also need to provide explanation
to the observations that are interesting or surprising.
Please also list the important code snippets followed by
explanation. Simply attaching code without any explanation will not
receive credits.

%%%%%%%%%%%%%%%%%%%%%%%%%%%%%%%%%%%%%%%%


\end{document}
%%%%%%%%%%%%%%%%%%%%%%%%%%%%%%%%%%%%%%%%%%%%%%%%%%%%%%%%
%%%%%%%%%%%%%%%%%%%%%%%%%%%%%%%%%%%%%%%%%%%%%%%%%%%%%%%%



