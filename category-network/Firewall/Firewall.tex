%%%%%%%%%%%%%%%%%%%%%%%%%%%%%%%%%%%%%%%%%%%%%%%%%%%%%%%%%%%%%%%%%%%%%%
%%  Copyright by Wenliang Du.                                       %%
%%  This work is licensed under the Creative Commons                %%
%%  Attribution-NonCommercial-ShareAlike 4.0 International License. %%
%%  To view a copy of this license, visit                           %%
%%  http://creativecommons.org/licenses/by-nc-sa/4.0/.              %%
%%%%%%%%%%%%%%%%%%%%%%%%%%%%%%%%%%%%%%%%%%%%%%%%%%%%%%%%%%%%%%%%%%%%%%

\documentclass[11pt]{article}

\usepackage[most]{tcolorbox}
\usepackage{times}
\usepackage{epsf}
\usepackage{epsfig}
\usepackage{amsmath, alltt, amssymb, xspace}
\usepackage{wrapfig}
\usepackage{fancyhdr}
\usepackage{url}
\usepackage{verbatim}
\usepackage{fancyvrb}
\usepackage{adjustbox}
\usepackage{listings}
\usepackage{color}
\usepackage{subfigure}
\usepackage{cite}
\usepackage{sidecap}
\usepackage{pifont}
\usepackage{mdframed}
\usepackage{textcomp}
\usepackage{enumitem}


% Horizontal alignment
\topmargin      -0.50in  % distance to headers
\oddsidemargin  0.0in
\evensidemargin 0.0in
\textwidth      6.5in
\textheight     8.9in 

\newcommand{\todo}[1]{
\vspace{0.1in}
\fbox{\parbox{6in}{TODO: #1}}
\vspace{0.1in}
}


\newcommand{\unix}{{\tt Unix}\xspace}
\newcommand{\linux}{{\tt Linux}\xspace}
\newcommand{\minix}{{\tt Minix}\xspace}
\newcommand{\ubuntu}{{\tt Ubuntu}\xspace}
\newcommand{\setuid}{{\tt Set-UID}\xspace}
\newcommand{\openssl} {\texttt{openssl}}


\pagestyle{fancy}
\lhead{\bfseries SEED Labs}
\chead{}
\rhead{\small \thepage}
\lfoot{}
\cfoot{}
\rfoot{}


\definecolor{dkgreen}{rgb}{0,0.6,0}
\definecolor{gray}{rgb}{0.5,0.5,0.5}
\definecolor{mauve}{rgb}{0.58,0,0.82}
\definecolor{lightgray}{gray}{0.90}


\lstset{%
  frame=none,
  language=,
  backgroundcolor=\color{lightgray},
  aboveskip=3mm,
  belowskip=3mm,
  showstringspaces=false,
%  columns=flexible,
  basicstyle={\small\ttfamily},
  numbers=none,
  numberstyle=\tiny\color{gray},
  keywordstyle=\color{blue},
  commentstyle=\color{dkgreen},
  stringstyle=\color{mauve},
  breaklines=true,
  breakatwhitespace=true,
  tabsize=3,
  columns=fullflexible,
  keepspaces=true,
  escapeinside={(*@}{@*)}
}

\newcommand{\newnote}[1]{
\vspace{0.1in}
\noindent
\fbox{\parbox{1.0\textwidth}{\textbf{Note:} #1}}
%\vspace{0.1in}
}


%% Submission
\newcommand{\seedsubmission}{You need to submit a detailed lab report, with screenshots,
to describe what you have done and what you have observed.
You also need to provide explanation
to the observations that are interesting or surprising.
Please also list the important code snippets followed by
explanation. Simply attaching code without any explanation will not
receive credits.}

%% Book
\newcommand{\seedbook}{\textit{Computer \& Internet Security: A Hands-on Approach}, 2nd
Edition, by Wenliang Du. See details at \url{https://www.handsonsecurity.net}.}

%% Videos
\newcommand{\seedisvideo}{\textit{Internet Security: A Hands-on Approach},
by Wenliang Du. See details at \url{https://www.handsonsecurity.net/video.html}.}

\newcommand{\seedcsvideo}{\textit{Computer Security: A Hands-on Approach},
by Wenliang Du. See details at \url{https://www.handsonsecurity.net/video.html}.}

%% Lab Environment
\newcommand{\seedenvironment}{This lab has been tested on our pre-built
Ubuntu 16.04 VM, which can be downloaded from the SEED website.}






\newcommand{\seedlabcopyright}[1]{
\vspace{0.1in}
\fbox{\parbox{6in}{\small Copyright \copyright\ {#1}\ \ by Wenliang Du.\\
      This work is licensed under a Creative Commons
      Attribution-NonCommercial-ShareAlike 4.0 International License.
      If you remix, transform, or build upon the material, 
      this copyright notice must be left intact, or reproduced in a way that is reasonable to
      the medium in which the work is being re-published.}}
\vspace{0.1in}
}





\newcommand{\telnet} {\texttt{telnet}\xspace}
\newcommand{\iptables}{\texttt{iptables}\xspace}
\newcommand{\netfilter}{\texttt{netfilter}\xspace}
\newcommand{\Netfilter}{\texttt{Netfilter}\xspace}

\newcommand{\firewallFigs}{./Figs}
\lhead{\bfseries SEED Labs -- Firewall Lab}

\begin{document}



\begin{center}
{\LARGE Firewall Lab}
\end{center}

\seedlabcopyright{2006 - 2020}



% *******************************************
% SECTION
% ******************************************* 
\section{Overview}

The learning objective of this lab is two-fold: learning
how firewalls work, and setting up a simple firewall
for a network. Students will first 
implement a simple stateless packet-filtering firewall, 
which inspects packets, and decides 
whether to drop or forward a packet based on firewall rules. 
Through this implementation task, students can get the 
basic ideas on how firewall works.


Actually, Linux already has a built-in firewall, also based on 
\texttt{netfilter}. This firewall is called \iptables. 
Students will be given a simple network topology, and are asked to
use \iptables to set up firewall rules to protect the network. 
Students will also be exposed to several other interesting 
applications of \iptables. 
This lab covers the following topics:


\begin{itemize}[noitemsep]
\item Firewall
\item Netfilter
\item Loadable kernel module
\item Using \iptables to set up firewall rules
\item Various applications of \iptables
\end{itemize}


\paragraph{Readings and videos.}
Detailed coverage of firewalls can be found in the following:

\begin{itemize}
\item Chapter 17 of the SEED Book, \seedbook
\item Section 9 of the SEED Lecture, \seedisvideo
\end{itemize}


\paragraph{Lab environment.} \seedenvironmentB




% *******************************************
% SECTION
% ******************************************* 
\section{Environment Setup}


We will write a separate document regarding environment
setup, for the common part, we refer students to 
read the document (read Section 1 and 3). 
In this document, we provide additional setup and 
background that are specific to this lab. 


% -------------------------------------------
% SUBSECTION
% ------------------------------------------- 
\subsection{Useful Commands}

For example, there are some commands that are more specific
to this lab, we will provide them here. 


\begin{lstlisting}
// Enable IP forwarding 
$ sudo sysctl ...
\end{lstlisting}
 



 




% *******************************************
% SECTION
% ******************************************* 
\section{Task 1: Implementing a Simple Firewall} 


In this task, we will implement a simple packet filtering 
type of firewall, which 
inspects each incoming and outgoing packets, and enforces the firewall policies 
set by the administrator. Since the packet 
processing is done within the kernel, the filtering must also be 
done within the kernel. Therefore, it seems that implementing such
a firewall requires us to modify the \linux kernel. In the past, 
this had to be done by modifying and rebuilding 
the kernel. The modern \linux 
operating systems provide several new mechanisms 
to facilitate the manipulation of packets without rebuilding
the kernel image. These two mechanisms are 
\textit{Loadable Kernel Module} (\texttt{LKM}) and \texttt{Netfilter}.



% -------------------------------------------
% SUBSECTION
% ------------------------------------------- 
\subsection{Task 1.a: Implement a Simple Kernel Module}


{\tt LKM} allows us to add a new module to the kernel at the runtime. 
This new module enables us to extend the functionalities of the kernel,
without rebuilding the kernel or even rebooting the computer. 
The packet filtering part of a firewall can be implemented as an LKM. 
In this task, we will get familiar with LKM.


The following is a simple loadable kernel module. It prints out 
\texttt{"Hello World!"} when the module is loaded; when the module
is removed from the kernel, it prints out \texttt{"Bye-bye World!"}.
The messages are not printed out on the screen; they are 
actually printed into the \texttt{/var/log/syslog} file. You can
use \texttt{"dmesg | tail -10"} to read the last 10 lines of 
the messages. 


\begin{lstlisting}
#include <linux/module.h>
#include <linux/kernel.h>

int init_module(void)
{
    printk(KERN_INFO "Hello World!\n");
    return 0;
}

void cleanup_module(void)
{
    printk(KERN_INFO "Bye-bye World!.\n");
}
\end{lstlisting}

We now need to create {\tt Makefile}, which includes the following
contents (the above program is named {\tt hello.c}). Then 
just type {\tt make}, and the above program will be compiled
into a loadable kernel module (when you copy and paste the following
into \texttt{Makefile}, make sure replace the spaces before the 
\texttt{make} commands with a tab).


\begin{lstlisting}
obj-m += hello.o

all:
        make -C /lib/modules/$(shell uname -r)/build M=$(PWD) modules

clean:
        make -C /lib/modules/$(shell uname -r)/build M=$(PWD) clean
\end{lstlisting}


Once the module is built by typing {\tt make}, you can use the following commands to 
load the module, list all modules, and remove the module. 
Also, you can use {\tt modinfo mymod.ko} to show information about a 
Linux Kernel module.

\begin{lstlisting}
$ sudo insmod mymod.ko        (inserting a module)
$ lsmod                       (list all modules)
$ sudo rmmod mymod.ko         (remove the module)
$ dmesg | tail -10            (check the messages)
\end{lstlisting}


\paragraph{Task.} Please compile and run this simple kernel module on 
your VM and show your running results in the lab report. 



% -------------------------------------------
% SUBSECTION
% ------------------------------------------- 
\subsection{Task 1.b: Implement a Simple Firewall Using \texttt{Netfilter}}  


In this task, we will write our packet filtering program
as an LKM, and then insert in into the packet processing path
inside the kernel. This cannot be easily done in the past before 
the \Netfilter was introduced into the \linux.

{\tt Netfilter} is designed to facilitate the manipulation of 
packets by authorized users. {\tt Netfilter} achieves this 
goal by implementing a number of {\em hooks} in the 
\linux kernel. These hooks are inserted into various places, 
including the packet incoming and outgoing paths. 
If we want to manipulate the incoming packets, we simply
need to connect our own programs (within LKM) to the 
corresponding hooks. Once an incoming packet arrives, 
our program will be invoked. Our program can decide 
whether this packet should be blocked or not; moreover,
we can also modify the packets in the program.


In this task, you need to use LKM and {\tt Netfilter} to implement
the packet filtering module.  This module will fetch 
the firewall policies from a data structure, and use the 
policies to decide whether packets should be blocked or not.
To make your life easier, so you can focus on the filtering part, 
the core of firewalls, we allow you to hardcode your firewall policies 
in the program. You should support at least three different 
rules. Guidelines on how to use \texttt{Netfilter} can be 
found in Chapter 17 of the SEED book.


Using {\tt Netfilter} is quite straightforward. All we need to do
is to hook our functions (in the kernel module) to the corresponding
{\tt Netfilter} hooks. Here we show an example (the code can
be downloaded from the lab's website):


\begin{lstlisting}
#include <linux/module.h>
#include <linux/kernel.h>
#include <linux/netfilter.h>
#include <linux/netfilter_ipv4.h>
#include <linux/ip.h>
#include <linux/tcp.h>


/* This is the structure we shall use to register our function */
static struct nf_hook_ops nfho;

/* This is the hook function itself */
unsigned int hook_func(void *priv, struct sk_buff *skb, 
                       const struct nf_hook_state *state)

{
    /* This is where you can inspect the packet contained in
       the structure pointed by skb, and decide whether to accept 
       or drop it. You can even modify the packet */
 
    struct iphdr *iph;
    struct tcphdr *tcph;
    
    iph = ip_hdr(skb);
    tcph = (void *)iph+iph->ihl*4;
 
    if (iph->protocol == IPPROTO_TCP && tcph->dest == htons(23)) {
      printk(KERN_INFO "Dropping telnet packet to %d.%d.%d.%d\n",
          ((unsigned char *)&iph->daddr)[0],
          ((unsigned char *)&iph->daddr)[1],
          ((unsigned char *)&iph->daddr)[2],
          ((unsigned char *)&iph->daddr)[3]);
      return NF_DROP;       /* Drop this packet */
    } else {
      return NF_ACCEPT;     /* Accept this packet */
    }
}

/* Initialization routine */
int init_module()
{   /* Fill in our hook structure */
    nfho.hook = hook_func;         /* Handler function */
    nfho.hooknum  = NF_INET_PRE_ROUTING; /* First hook for IPv4 */
    nfho.pf       = PF_INET;
    nfho.priority = NF_IP_PRI_FIRST;   /* Make our function first */

    // nf_register_hook(&nfho);               // For Ubuntu 16.04 VM
    nf_register_net_hook(&init_net, &nfho);   // For Ubuntu 20.04 VM
    return 0;
}

/* Cleanup routine */
void cleanup_module()
{
    // nf_unregister_hook(&nfho);             // For Ubuntu 16.04 VM
    nf_unregister_net_hook(&init_net, &nfho); // For Ubuntu 20.04 VM
}
\end{lstlisting}


\paragraph{Note for Ubuntu 20.04 VM:}
The code in the SEED book was developed in Ubuntu 16.04. It needs to be changed slightly 
to work in Ubuntu 20.04. The change is in the hook registration and 
unregistration APIs. See the difference in the following:


\begin{lstlisting}
// Hook registration:
  nf_register_hook(&nfho);                  // For Ubuntu 16.04 VM
  nf_register_net_hook(&init_net, &nfho);   // For Ubuntu 20.04 VM


// Hook unregistration:
  nf_unregister_hook(&nfho);                // For Ubuntu 16.04 VM
  nf_unregister_net_hook(&init_net, &nfho); // For Ubuntu 20.04 VM
\end{lstlisting}
 

\paragraph{Tasks.} Based on the code provided above (they can be downloaded
from the lab's website), please do the following tasks:

\begin{itemize}
\item Compile and run the provided code. Based on your execution results,
explain the purpose of the firewall.  

\item The following are \netfilter hooks. Please replace the 
hook used in the sample code with each one of the following,
and describe your observation.

\begin{lstlisting}
   NF_INET_PRE_ROUTING,
   NF_INET_LOCAL_IN,
   NF_INET_FORWARD,
   NF_INET_LOCAL_OUT,
   NF_INET_POST_ROUTING,
\end{lstlisting}
 
\end{itemize}
 



% *******************************************
% SECTION
% ******************************************* 
\section{Task 2: Setting Up a Firewall}

In the previous task, we had a chance to build a simple firewall using \netfilter. Actually,
\linux already has a built-in firewall, also based on \netfilter. This firewall is called
\iptables. Technically, the kernel part implementation of the firewall
is called \texttt{Xtables}, while \iptables is a user-space program to
configure the firewall. However, \iptables is often used to refer to both the kernel-part
implementation and the user-space program. 



% -------------------------------------------
% SUBSECTION
% ------------------------------------------- 
\subsection{Background of \iptables}

In this task, we will use \iptables to set up a firewall. 
The \iptables firewall is designed not only to filter packets, but also to make changes to
packets. To help manage these firewall rules for different purposes, \iptables organizes all
rules using a hierarchical structure: table, chain, and rules.
There are several tables, each specifying the main purpose of the rules as shown
in Table~\ref{firewall:table:iptables}.
For example, rules for packet filtering should be
placed in the \texttt{filter} table, while rules for making changes to packets should be placed
in the \texttt{nat} or \texttt{mangle} tables.

Each table contains several chains, each of which corresponds to a \netfilter hook. Basically,
each chain indicates where its rules are enforced. For example, rules on
the \texttt{FORWARD} chain are enforced at the \texttt{NF\_INET\_FORWARD} hook, and rules on
the \texttt{INPUT} chain are enforced at the  \texttt{NF\_INET\_LOCAL\_IN} hook.

Each chain contains a set of firewall rules that will be enforced.
When we set up firewalls, we add rules to these chains.
For example, if we would like to block all incoming \telnet traffic, we would
add a rule to the \texttt{INPUT} chain of the \texttt{filter} table.  If we
would like to redirect all incoming \telnet traffic to a different
port on a different host, basically doing port forwarding, we can add a rule to the
\texttt{INPUT} chain of the \texttt{mangle} table, as we need to make changes to packets.


\begin{table}[htb]
        \centering
%       \renewcommand{\arraystretch}{1.2}
        \caption{\iptables Tables and Chains}
        \label{firewall:table:iptables}
        \centering

        \begin{tabular}{|l|l|l|}
                \hline
                \bfseries Table & \bfseries Chain & \bfseries Functionality \\
                \hline\hline
                filter          &    \texttt{INPUT}      & Packet filtering \\
                                &    \texttt{FORWARD}    & \\
                                &    \texttt{OUTPUT}      & \\
                \hline
                nat             &   \texttt{PREROUTING}    & Modifying source or destination \\
                                &   \texttt{INPUT}      & network addresses \\
                                &   \texttt{OUTPUT}      & \\
                                &   \texttt{POSTROUTING}   & \\
                \hline
                mangle          &   \texttt{PREROUTING}    & Packet content modification \\
                                &   \texttt{INPUT}      & \\
                                &   \texttt{FORWARD}     & \\
                                &   \texttt{OUTPUT}      & \\
                                &   \texttt{POSTROUTING}   & \\
                \hline
        \end{tabular}
\end{table}


% -------------------------------------------
% SUBSECTION
% ------------------------------------------- 
\subsection{Using \iptables}


To add rules to the chains in each table, we use the \iptables command,
which is a quite powerful command. 
Students can find the manual of \iptables by typing \texttt{"man iptables"} 
or easily find many tutorials from online. 
What makes \iptables complicated is the many command-line arguments 
that we need to provide when
using the command. However, 
if we understand the structure of these command-line arguments, 
we will find out that the command is not that complicated. 


In a typical \iptables command, we add a rule to or remove a rule 
from one of the chains in one of the tables, so we need to 
specify a table name (the default is \texttt{filter}), a chain name, 
and an operation on the chain. After that, we specify the rule, which
is basically a pattern that will be matched with each of the 
packets passing through. If there is a match, an action will be 
performed on this packet. 
The general structure of the command is depicted in the following:

\begin{lstlisting}
# iptables -t <table> -<operation> <chain>  <rule>   -j <target>
           ---------- --------------------  -------  -----------
              Table          Chain           Rule      Action
\end{lstlisting}


The rule is the most complicated part of the \iptables command. 
We will provide additional information later on when we use 
specific rules. In the following, we list some commonly 
used commands: 


\begin{lstlisting}
// List all the rules in a table (without line number)
$ sudo iptables -t nat -L

// List all the rules in a table (with line number)
$ sudo iptables -t filter -L --line-numbers

// Delete rule No. 2 in the INPUT chain of the filter table 
$ sudo iptables -t filter -D INPUT 2

// Drop all the incoming packets that satisfy the <rule>
$ sudo iptables -t filter -A INPUT <rule>  -j DROP
\end{lstlisting}


\paragraph{Note.} Docker relies on \iptables to manage 
the networks it creates, so it adds many rules to \iptables.
When we manipulate \iptables rules, we should be careful 
not to remove Docker rules. For example, it will be quite
dangerous to run the \texttt{"iptables -F"} command, because 
it removes all the rules in the \texttt{filter} table,
including many of the Docker rules. That will cause 
trouble to Docker containers. 


% -------------------------------------------
% SUBSECTION
% ------------------------------------------- 
\subsection{Task 2.a: Set Up a Stateless Firewall}

In this task, we will set up firewall rules on the router container
to protect the \texttt{192.168.60.0/24} network. We will only use 
the \texttt{filter} table in this task.  
Your firewall should satisfy the following requirements. 
In your report, you should provide screenshots to demonstrate 
that your firewall does satisfy the requirements.


\begin{itemize}
\item Ingress: Prevent outsiders from accessing any internal ports (including
both TCP and UDP), other than port TCP 80, which is used by web servers. 
We have installed a very simple web server in the container. You can
add the following command to the command entry of a container
in the Compose file if you want to run the server on that container:

\begin{lstlisting}
\end{lstlisting}
 
\item Ingress: Your firewall should allow DNS replies to come back. 

\item Ingress: Block all the ICMP echo request from outside, but 
do allow the ICMP reply packets to come in; otherwise, internal 
computers will not be able to run \texttt{ping}.  

\item Egress: Prevent internal machines from connecting to the outside via \telnet.

\item Egress: Prevent internal machines from visiting an external web site. 
You can choose any web site that you like to block, but keep in mind, 
some web servers have multiple IP addresses. 

\end{itemize}


To see what options that we can use for rules specific to the tcp, udp, and 
icmp protocols, we can use the following commands. 

\begin{lstlisting}
$ iptables -p tcp  -h
$ iptables -p udp  -h
$ iptables -p icmp -h
\end{lstlisting}
 


% -------------------------------------------
% SUBSECTION
% ------------------------------------------- 
\subsection{Task 2.b: Experiment with the Connection Tracking} 


In the previous task, we have only set up stateless firewalls, which inspect each
packet independently. However, packets
are usually not independent; they may be part of a TCP connection,
or they may be ICMP packets triggered by other packets. Treating them
independently does not take into consideration the context of the
packets, and can thus lead to inaccurate or unsafe firewall rules.
For example, if we would like to allow TCP packets to get into our network
only if a connection was made first, we cannot achieve that easily 
using stateless packet filters, because when the firewall examines each individual TCP packet,
it has no idea about whether the packet belongs to an existing connection
or not, unless the firewall maintains some state information for each connection.
If it does that, it becomes a stateful firewall.


To support stateful firewalls, we need to be able to track connections. 
This is achieved by the \texttt{conntrack} mechanism inside the kernel. 
In this task, we will conduct experiments related to this module, and 
get familiar with the connection tracking mechanism. 
In our experiment, we will check the connection tracking information
on the router container. This can be done using the following command: 

\begin{lstlisting}
# conntrack -L
\end{lstlisting}

The goal of the task is to use a series of experiments to 
help students understand the 
connection concept in this tracking mechanism, especially
for the ICMP and UDP protocols, because unlike TCP,  they 
do not have connections. 
Before doing the experiments, we need to restart the server container to
remove all the firewall rules added in the previous task. We can either 
manually remove those rules, or simply restart the router container 
using the following commands:

\begin{lstlisting}
$ docker container  ls
7e234511e597  firewall-router
$ docker restart 7e
\end{lstlisting}

Please conduct the following experiments. For each experiment, please 
describe your observation, along with your explanation. 

\begin{itemize}
\item ICMP experiment: Run the following command and 
check the connection tracking information on the router. Describe
your observation. How long is the ICMP connection state be kept? 

\begin{lstlisting}
// On 192.168.60.5, send out ICMP packets
# ping 10.9.0.5
\end{lstlisting}

\item UDP experiment: Run the following command and 
check the connection tracking information on the router. Describe
your observation. How long is the UDP connection state be kept? 


\begin{lstlisting}
// On 10.9.0.5, start a netcat UDP server
# nc -lnuv 9090

// On 192.168.60.5, send out UDP packets 
# nc -u 10.9.0.5 9090
\end{lstlisting}


\item TCP experiment: Run the following command and 
check the connection tracking information on the router. Describe
your observation. How long is the TCP connection state be kept? 

\begin{lstlisting}
// On 10.9.0.5, start a netcat UDP server
# nc -lnv 9090

// On 192.168.60.5, send out TCP packets 
# nc 10.9.0.5 9090
\end{lstlisting}

\end{itemize}
 


% -------------------------------------------
% SUBSECTION
% ------------------------------------------- 
\subsection{Task 2.c: Setting Up a Stateful Firewall} 


Now we are ready to set up firewall rules based on connections. 
In the following example, 
the \texttt{"-m conntrack"} option indicates that we are using the \texttt{conntrack} module,
which is a very important module for \iptables; it tracks connections, and
\iptables replies on the tracking information to build stateful firewalls. 
The \texttt{--ctsate ESTABLISHED,RELATED} indicates that whether a packet
belongs to an \texttt{ESTABLISHED} or \texttt{RELATED} connection.

\begin{lstlisting}
# sudo iptables -A OUTPUT -p tcp -m conntrack
                --ctstate ESTABLISHED,RELATED -j ACCEPT
\end{lstlisting}


Please improve the firewall from Task 2.c using the \texttt{conntrack} module, 
so the firewall will allow TCP, UDP, and ICMP packets to go through (both direction)
if they belong to an existing connection. 

 



% *******************************************
% SECTION
% ******************************************* 
\section{Task 3: Other Applications of \iptables}

Before doing each subtask in this task, please 
restart the server container, so the setting from the 
previous task/subtask won't affect the current task. 


\begin{lstlisting}
$ docker container  ls
7e234511e597  firewall-router
$ docker restart 7e
\end{lstlisting}



% -------------------------------------------
% SUBSECTION
% ------------------------------------------- 
\subsection{Task 3.a: Packet Limiting}


Give the command, ask students to show the experiment results. 

\begin{lstlisting}
# iptables -A INPUT -p icmp -m limit \
              --limit 10/min --limit-burst 5 -j ACCEPT

# iptables -A INPUT -p icmp -j DROP
\end{lstlisting}
 

% -------------------------------------------
% SUBSECTION
% ------------------------------------------- 
\subsection{Task 3.b: Port Forwarding}


Add the following to the Compose file for each 
\begin{lstlisting}
command: bash -c "
                      ...
                      nc -lnkuv 9090 &&
                      /start.sh
                 "
\end{lstlisting}



% -------------------------------------------
% SUBSECTION
% ------------------------------------------- 
\subsection{Task 3.c: Load Balancing (verified)}

The order matters. How chains are traversed. 
 


Students should decide the values (p1=0.33, p2=0.5, p3=1)

\begin{lstlisting}
iptables -t nat -A PREROUTING -p udp --dport 9090  \
         -m statistic --mode random --probability .33     \
         -j DNAT --to-destination 192.168.60.5:9090

# iptables -t nat -A PREROUTING -p tcp --dport 8000  \
              -m statistic --mode random --probability <p1>       \
              -j DNAT --to-destination <host 1>:8000

# iptables -t nat -A PREROUTING -p tcp --dport 8000  \
              -m statistic --mode random --probability <p2>       \
              -j DNAT --to-destination <host 2>:8000

# iptables -t nat -A PREROUTING -p tcp --dport 8000  \
              -m statistic --mode random --probability <p3>       \
              -j DNAT --to-destination <host 3>:8000
\end{lstlisting}



From the VM  
\begin{lstlisting}
echo asdfsdafsadf | nc -u 192.168.60.11 9090
\end{lstlisting}
 

\paragraph{Generalization.} What if we have \texttt{K} hosts?  


 

% -------------------------------------------
% SUBSECTION
% ------------------------------------------- 
\subsection{Task 3.c: Changing TTL (verified)}



Limit how far a packet can go. 


\begin{lstlisting}
# iptables -j TTL -h
...
TTL target options
  --ttl-set value		Set TTL to <value 0-255>
  --ttl-dec value		Decrement TTL by <value 1-255>
  --ttl-inc value		Increment TTL by <value 1-255>


# iptables -t mangle -A POSTROUTING -j TTL --ttl-set 1

# iptables -t <table> -A <chain> -j TTL --<ttl command> <ttl value>

\end{lstlisting}
 




% *******************************************
% SECTION
% ******************************************* 
\section{Task 4: Reverse Path Filter}



\begin{lstlisting}

\end{lstlisting}
 






% *******************************************
% SECTION
% ******************************************* 
\section{Submission and Demonstration}


\seedsubmission



\end{document}
%%%%%%%%%%%%%%%%%%%%%%%%%%%%%%%%%%%%%%%%%%%%%%%%%%%%%%%%
%%%%%%%%%%%%%%%%%%%%%%%%%%%%%%%%%%%%%%%%%%%%%%%%%%%%%%%%



