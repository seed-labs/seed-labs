%%%%%%%%%%%%%%%%%%%%%%%%%%%%%%%%%%%%%%%%%%%%%%%%%%%%%%%%%%%%%%%%%%%%%%
%%  Copyright by Wenliang Du.                                       %%
%%  This work is licensed under the Creative Commons                %%
%%  Attribution-NonCommercial-ShareAlike 4.0 International License. %%
%%  To view a copy of this license, visit                           %%
%%  http://creativecommons.org/licenses/by-nc-sa/4.0/.              %%
%%%%%%%%%%%%%%%%%%%%%%%%%%%%%%%%%%%%%%%%%%%%%%%%%%%%%%%%%%%%%%%%%%%%%%

\newcommand{\commonfolder}{../../common-files}

\documentclass[11pt]{article}

\usepackage[most]{tcolorbox}
\usepackage{times}
\usepackage{epsf}
\usepackage{epsfig}
\usepackage{amsmath, alltt, amssymb, xspace}
\usepackage{wrapfig}
\usepackage{fancyhdr}
\usepackage{url}
\usepackage{verbatim}
\usepackage{fancyvrb}
\usepackage{adjustbox}
\usepackage{listings}
\usepackage{color}
\usepackage{subfigure}
\usepackage{cite}
\usepackage{sidecap}
\usepackage{pifont}
\usepackage{mdframed}
\usepackage{textcomp}
\usepackage{enumitem}


% Horizontal alignment
\topmargin      -0.50in  % distance to headers
\oddsidemargin  0.0in
\evensidemargin 0.0in
\textwidth      6.5in
\textheight     8.9in 

\newcommand{\todo}[1]{
\vspace{0.1in}
\fbox{\parbox{6in}{TODO: #1}}
\vspace{0.1in}
}


\newcommand{\unix}{{\tt Unix}\xspace}
\newcommand{\linux}{{\tt Linux}\xspace}
\newcommand{\minix}{{\tt Minix}\xspace}
\newcommand{\ubuntu}{{\tt Ubuntu}\xspace}
\newcommand{\setuid}{{\tt Set-UID}\xspace}
\newcommand{\openssl} {\texttt{openssl}}


\pagestyle{fancy}
\lhead{\bfseries SEED Labs}
\chead{}
\rhead{\small \thepage}
\lfoot{}
\cfoot{}
\rfoot{}


\definecolor{dkgreen}{rgb}{0,0.6,0}
\definecolor{gray}{rgb}{0.5,0.5,0.5}
\definecolor{mauve}{rgb}{0.58,0,0.82}
\definecolor{lightgray}{gray}{0.90}


\lstset{%
  frame=none,
  language=,
  backgroundcolor=\color{lightgray},
  aboveskip=3mm,
  belowskip=3mm,
  showstringspaces=false,
%  columns=flexible,
  basicstyle={\small\ttfamily},
  numbers=none,
  numberstyle=\tiny\color{gray},
  keywordstyle=\color{blue},
  commentstyle=\color{dkgreen},
  stringstyle=\color{mauve},
  breaklines=true,
  breakatwhitespace=true,
  tabsize=3,
  columns=fullflexible,
  keepspaces=true,
  escapeinside={(*@}{@*)}
}

\newcommand{\newnote}[1]{
\vspace{0.1in}
\noindent
\fbox{\parbox{1.0\textwidth}{\textbf{Note:} #1}}
%\vspace{0.1in}
}


%% Submission
\newcommand{\seedsubmission}{You need to submit a detailed lab report, with screenshots,
to describe what you have done and what you have observed.
You also need to provide explanation
to the observations that are interesting or surprising.
Please also list the important code snippets followed by
explanation. Simply attaching code without any explanation will not
receive credits.}

%% Book
\newcommand{\seedbook}{\textit{Computer \& Internet Security: A Hands-on Approach}, 2nd
Edition, by Wenliang Du. See details at \url{https://www.handsonsecurity.net}.}

%% Videos
\newcommand{\seedisvideo}{\textit{Internet Security: A Hands-on Approach},
by Wenliang Du. See details at \url{https://www.handsonsecurity.net/video.html}.}

\newcommand{\seedcsvideo}{\textit{Computer Security: A Hands-on Approach},
by Wenliang Du. See details at \url{https://www.handsonsecurity.net/video.html}.}

%% Lab Environment
\newcommand{\seedenvironment}{This lab has been tested on our pre-built
Ubuntu 16.04 VM, which can be downloaded from the SEED website.}






\newcommand{\seedlabcopyright}[1]{
\vspace{0.1in}
\fbox{\parbox{6in}{\small Copyright \copyright\ {#1}\ \ by Wenliang Du.\\
      This work is licensed under a Creative Commons
      Attribution-NonCommercial-ShareAlike 4.0 International License.
      If you remix, transform, or build upon the material, 
      this copyright notice must be left intact, or reproduced in a way that is reasonable to
      the medium in which the work is being re-published.}}
\vspace{0.1in}
}






\lhead{\bfseries SEED Labs -- BGP Exploration and Attack}
\newcommand{\bgpFigs}{./Figs}

\begin{document}


\begin{center}
{\LARGE BGP Exploration and Attack Lab}

  \vspace{0.1in}
  {\LARGE (Work in Progress)}
\end{center}

\seedlabcopyright{2021}



% *******************************************
% SECTION
% ******************************************* 
\section{Overview}

Border Gateway Protocol (BGP) is the standard exterior gateway protocol
designed to exchange routing and reachability information among autonomous
systems (AS) on the Internet. It is the ``glue'' of the Internet,
and is an essential piece of the Internet infrastructure. It is 
also a primary attack target, because if attackers can 
compromise BGP, they can disconnect the Internet and redirect traffics. 

Because of the complexity of BGP, it is hard to do everything in a single lab. 
Therefore, we have developed a series of labs related to BGP. This lab
is the first in the series, and it is the basis for all the other BGP labs.  
The goal of this lab is to help students understand how
BGP ``glues'' the Internet together, and how the Internet is actually
connected. In this lab, we guide students to build a small-scale Internet.
We call this Internet the Internet Simulator (or simply
Simulator in short). This simulator will be the basis for 
all other BGP labs, as well as for some non-BGP labs that 
depends on the Internet. 
This lab covers the following topics:
\begin{itemize}[noitemsep]
\item How the BGP protocol works
\item BGP configuration
\item Routing 
\item Internet Exchange Point (IXP)
\item BGP attack
\end{itemize}


\paragraph{Videos.}
Detailed coverage of the BGP protocol can be found in 
Section 10 of the SEED Lecture at Udemy, \seedisvideo 
The lecture was recorded before this lab was developed; 
it focuses mostly on the theory part, i.e., explaining how the BGP protocol works. 
This lab provides the practical part.  


\paragraph{Lab environment.} 
\seedenvironmentB
\nodependency


\paragraph{Acknowledgment.} 
This lab was developed with the help of Honghao Zeng, a graduate student
in the Department of Electrical Engineering and Computer Science at Syracuse University.
The SEED project is funded by the US National Science Foundation. 


% *******************************************
% SECTION
% *******************************************
\section{The Internet Emulator} 

This lab will be performed inside the SEED Internet Emulator. We will
provide a pre-built emulator in two different forms: (1) Python code,
and (2) container files. The container files are generated from
the Python code. However, students need to install the SEED Emulator source 
code from the GitHub to run the Python code. The container files
can be directly used without the Emulator source code. 


% *******************************************
% SECTION
% *******************************************
\section{EBGP} 

Pick a BGP router from the emulator. 
Design the following activities.

\begin{itemize}
  \item Start with a stub AS (with two upstream ASes)
  \item Study the BGP configuration file, and understand its entries
  \item Explain the relationships among all the tables, pipe, etc.
  \item Inspect the BGP route table 
  \item Disable/Enable BGP sessions
  \item Simple filter  
  \item Path selection (use the filter to change the other attributes,
    see how it affects the path selection).
\end{itemize}


\paragraph{Advanced features.} 
These are advanced features. Not sure whether they should be included or not.
We temporarily list them here.

\begin{itemize}
  \item Community: label for the route
\end{itemize}
 


% *******************************************
% SECTION
% *******************************************
\section{Interior BGP (IBGP)} 


\begin{itemize}
  \item Explain the IBGP setup in the configuration 
  \item Inspect the route created by IBGP
  \item Disable OSPF and see the result
\end{itemize}
 


% *******************************************
% SECTION
% *******************************************
\section{IP Anycast} 

Provide the AS (IP address T is an IP anycast address). 
Ask students to conduct the following experiment:

\begin{itemize}
  \item Inspect the BGP route table to identify the route.
  \item From A and B, ping T. They get to two different T.
  \item Break a BGP session, so they get to the same T.
  \item Use UDP instead of ping.
\end{itemize}
 


% *******************************************
% SECTION
% *******************************************
\section{BGP Attacks} 


% -------------------------------------------
% SUBSECTION
% -------------------------------------------
\subsection{BGP Session Reset Attack} 



Countermeasure: TTL security switch (RFC 5082).
Design an experiment to demonstrate its effect.

\begin{lstlisting}
protocol bgp c_as12 {
    ipv4 {
        table t_bgp;
        import filter {
	   ...
        };
        export all;
        next hop self;
    };
    local 10.104.0.3 as 3;
    neighbor 10.104.0.12 as 12;
    ttl security;
}
\end{lstlisting}
 

% -------------------------------------------
% SUBSECTION
% -------------------------------------------
\subsection{IP Prefix Hijacking} 



\begin{lstlisting}
protocol static hijacks {
    ipv4 {
        table t_bgp;
    };
    route 10.153.0.0/25 blackhole   { bgp_large_community.add(LOCAL_COMM); };
    route 10.153.0.128/25 blackhole { bgp_large_community.add(LOCAL_COMM); };
}
\end{lstlisting}
 


% *******************************************
% SECTION
% ******************************************* 
\section{Submission}

%%%%%%%%%%%%%%%%%%%%%%%%%%%%%%%%%%%%%%%%

You need to submit a detailed lab report, with screenshots,
to describe what you have done and what you have observed.
You also need to provide explanation
to the observations that are interesting or surprising.
Please also list the important code snippets followed by
explanation. Simply attaching code without any explanation will not
receive credits.

%%%%%%%%%%%%%%%%%%%%%%%%%%%%%%%%%%%%%%%%


\end{document}



