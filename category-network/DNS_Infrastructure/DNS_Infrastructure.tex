%%%%%%%%%%%%%%%%%%%%%%%%%%%%%%%%%%%%%%%%%%%%%%%%%%%%%%%%%%%%%%%%%%%%%%
%%  Copyright by Wenliang Du.                                       %%
%%  This work is licensed under the Creative Commons                %%
%%  Attribution-NonCommercial-ShareAlike 4.0 International License. %%
%%  To view a copy of this license, visit                           %%
%%  http://creativecommons.org/licenses/by-nc-sa/4.0/.              %%
%%%%%%%%%%%%%%%%%%%%%%%%%%%%%%%%%%%%%%%%%%%%%%%%%%%%%%%%%%%%%%%%%%%%%%

\newcommand{\commonfolder}{../../common-files}

\documentclass[11pt]{article}

\usepackage[most]{tcolorbox}
\usepackage{times}
\usepackage{epsf}
\usepackage{epsfig}
\usepackage{amsmath, alltt, amssymb, xspace}
\usepackage{wrapfig}
\usepackage{fancyhdr}
\usepackage{url}
\usepackage{verbatim}
\usepackage{fancyvrb}
\usepackage{adjustbox}
\usepackage{listings}
\usepackage{color}
\usepackage{subfigure}
\usepackage{cite}
\usepackage{sidecap}
\usepackage{pifont}
\usepackage{mdframed}
\usepackage{textcomp}
\usepackage{enumitem}


% Horizontal alignment
\topmargin      -0.50in  % distance to headers
\oddsidemargin  0.0in
\evensidemargin 0.0in
\textwidth      6.5in
\textheight     8.9in 

\newcommand{\todo}[1]{
\vspace{0.1in}
\fbox{\parbox{6in}{TODO: #1}}
\vspace{0.1in}
}


\newcommand{\unix}{{\tt Unix}\xspace}
\newcommand{\linux}{{\tt Linux}\xspace}
\newcommand{\minix}{{\tt Minix}\xspace}
\newcommand{\ubuntu}{{\tt Ubuntu}\xspace}
\newcommand{\setuid}{{\tt Set-UID}\xspace}
\newcommand{\openssl} {\texttt{openssl}}


\pagestyle{fancy}
\lhead{\bfseries SEED Labs}
\chead{}
\rhead{\small \thepage}
\lfoot{}
\cfoot{}
\rfoot{}


\definecolor{dkgreen}{rgb}{0,0.6,0}
\definecolor{gray}{rgb}{0.5,0.5,0.5}
\definecolor{mauve}{rgb}{0.58,0,0.82}
\definecolor{lightgray}{gray}{0.90}


\lstset{%
  frame=none,
  language=,
  backgroundcolor=\color{lightgray},
  aboveskip=3mm,
  belowskip=3mm,
  showstringspaces=false,
%  columns=flexible,
  basicstyle={\small\ttfamily},
  numbers=none,
  numberstyle=\tiny\color{gray},
  keywordstyle=\color{blue},
  commentstyle=\color{dkgreen},
  stringstyle=\color{mauve},
  breaklines=true,
  breakatwhitespace=true,
  tabsize=3,
  columns=fullflexible,
  keepspaces=true,
  escapeinside={(*@}{@*)}
}

\newcommand{\newnote}[1]{
\vspace{0.1in}
\noindent
\fbox{\parbox{1.0\textwidth}{\textbf{Note:} #1}}
%\vspace{0.1in}
}


%% Submission
\newcommand{\seedsubmission}{You need to submit a detailed lab report, with screenshots,
to describe what you have done and what you have observed.
You also need to provide explanation
to the observations that are interesting or surprising.
Please also list the important code snippets followed by
explanation. Simply attaching code without any explanation will not
receive credits.}

%% Book
\newcommand{\seedbook}{\textit{Computer \& Internet Security: A Hands-on Approach}, 2nd
Edition, by Wenliang Du. See details at \url{https://www.handsonsecurity.net}.}

%% Videos
\newcommand{\seedisvideo}{\textit{Internet Security: A Hands-on Approach},
by Wenliang Du. See details at \url{https://www.handsonsecurity.net/video.html}.}

\newcommand{\seedcsvideo}{\textit{Computer Security: A Hands-on Approach},
by Wenliang Du. See details at \url{https://www.handsonsecurity.net/video.html}.}

%% Lab Environment
\newcommand{\seedenvironment}{This lab has been tested on our pre-built
Ubuntu 16.04 VM, which can be downloaded from the SEED website.}






\newcommand{\seedlabcopyright}[1]{
\vspace{0.1in}
\fbox{\parbox{6in}{\small Copyright \copyright\ {#1}\ \ by Wenliang Du.\\
      This work is licensed under a Creative Commons
      Attribution-NonCommercial-ShareAlike 4.0 International License.
      If you remix, transform, or build upon the material, 
      this copyright notice must be left intact, or reproduced in a way that is reasonable to
      the medium in which the work is being re-published.}}
\vspace{0.1in}
}




\hypersetup{%
    pdfborder = {0 0 0}
}

\lhead{\bfseries SEED Labs -- DNS Infrastructure}
\newcommand{\dnsFigs}{./Figs}


\usepackage{hyperref}

\begin{document}


\begin{center}
{\LARGE DNS Infrastructure Lab}
\end{center}

\seedlabcopyright{2021}



% *******************************************
% SECTION
% ******************************************* 
\section{Overview}

DNS (Domain Name System) is the Internet's phone book; it
translates hostnames to IP addresses (and vice versa).
This translation is through DNS resolution, which happens behind
the scene. The resolution process involves many nameservers,
including root servers, TLD servers, and final domain servers.
These nameservers form the entire DNS system, which is an
essential infrastructure for the Internet.

To help students understand how these nameservers work together
to form the infrastructure, we will create a miniature DNS system
called \textit{DNS in a Box}. As suggested by its name,
the entire DNS system, which consists of multiple
nameservers, runs inside a single machine. This is made
possible by the container technology.


Even though this system is small, it has all the essential
elements of a real DNS infrastructure. By building such a system,
students will have a deeper understanding of how the DNS actually works.
Although this lab is not a security lab, it is the basis for
several SEED labs. This lab covers the following topics:

\begin{itemize}[noitemsep]
\item DNS and how it works
\item The DNS query process
\item Root and TLD servers
\item Docker container, docker compose
\end{itemize}


\paragraph{Readings and videos.}
Detailed coverage of the DNS protocol can be found in the following:

\begin{itemize}
\item Chapter 18 of the SEED Book, \seedbook
\item Section 7 of the SEED Lecture, \seedisvideo
\end{itemize}


\paragraph{Lab environment.} 
\seedenvironmentB
\nodependency


% -------------------------------------------
% SUBSECTION
% -------------------------------------------
\section{The Lab Setup and the SEED Internet Emulator}

This lab will be performed inside the SEED Internet Emulator (simply
called the emulator in this document).
We provide a pre-built emulator in two different forms: Python code
and container files. The container files are generated from
the Python code, but students need to install the SEED Emulator source
code from the GitHub to run the Python code. The container files
can be directly used without the emulator source code.
Instructors who would like to customize the emulator can modify the Python
code, generate their own container files, and then provide the
files to students, replacing the ones included in the
lab setup file.


\paragraph{Download the emulator files.}
Please download the \texttt{Labsetup.zip} file from the web page, and
unzip it. The files inside the \texttt{output/} sub-folder are the actual
emulation files (container files) that are
generated from the Python code \texttt{internet-emulator.py}.


\paragraph{Start the emulation.}
We will directly use the container files in the \texttt{output/} folder.
Go to this folder, and run the following docker commands
to build and start the containers. We recommend that you run the emulator inside
the provided SEED Ubuntu 20.04 VM, but doing it in a generic Ubuntu 20.04 operating system
should not have any problem, as long as the docker software is installed.
Readers can find the docker manual from
\href{https://github.com/seed-labs/seed-labs/blob/master/manuals/docker/SEEDManual-Container.md}
{\underline{this link}}.

\begin{lstlisting}
$ docker-compose build
$ docker-compose up

// Aliases for the Compose commands above (only available in the SEED VM)
$ dcbuild       # Alias for: docker-compose build
$ dcup          # Alias for: docker-compose up
$ dcdown        # Alias for: docker-compose down
\end{lstlisting}


% *******************************************
% SECTION
% *******************************************
\section{Task 1: Local DNS Server and Public DNS Resolvers} 

Set up the local DNS server for all except AS-150. So students
can configure this one.
Set up the root hint.




% *******************************************
% SECTION
% *******************************************
\section{Task 2: Root and TLD Servers} 


% -------------------------------------------
% SUBSECTION
% -------------------------------------------
\subsection{The Root Servers} 



\begin{lstlisting}
$ dig @a.root-servers.net www.example.com
\end{lstlisting}
 




% -------------------------------------------
% SUBSECTION
% -------------------------------------------
\subsection{The TLD Servers} 



% *******************************************
% SECTION
% *******************************************
\section{Task 3: Domain Nameservers} 




% *******************************************
% SECTION
% *******************************************
\section{Task 4: Reverse DNS Lookup} 




% *******************************************
% SECTION
% *******************************************
\section{Task 5: IP Anycast}







% *******************************************
% SECTION
% *******************************************
%\section{Task 6: Using DNS for Load Balancing} 



% *******************************************
% SECTION
% ******************************************* 
\section{Submission}

%%%%%%%%%%%%%%%%%%%%%%%%%%%%%%%%%%%%%%%%

You need to submit a detailed lab report, with screenshots,
to describe what you have done and what you have observed.
You also need to provide explanation
to the observations that are interesting or surprising.
Please also list the important code snippets followed by
explanation. Simply attaching code without any explanation will not
receive credits.

%%%%%%%%%%%%%%%%%%%%%%%%%%%%%%%%%%%%%%%%



% *******************************************
% SECTION
% *******************************************
\section*{Acknowledgment} 

This lab was developed with the help of Honghao Zeng, 
a graduate student in the Department of Electrical Engineering 
and Computer Science at Syracuse University. 
The SEED project was funded in part 
by the grants from the US National Science Foundation
and Syracuse University.

\end{document}



