%%%%%%%%%%%%%%%%%%%%%%%%%%%%%%%%%%%%%%%%%%%%%%%%%%%%%%%%%%%%%%%%%%%%%%
%%  Copyright by Wenliang Du.                                       %%
%%  This work is licensed under the Creative Commons                %%
%%  Attribution-NonCommercial-ShareAlike 4.0 International License. %%
%%  To view a copy of this license, visit                           %%
%%  http://creativecommons.org/licenses/by-nc-sa/4.0/.              %%
%%%%%%%%%%%%%%%%%%%%%%%%%%%%%%%%%%%%%%%%%%%%%%%%%%%%%%%%%%%%%%%%%%%%%%

\documentclass[11pt]{article}

\usepackage[most]{tcolorbox}
\usepackage{times}
\usepackage{epsf}
\usepackage{epsfig}
\usepackage{amsmath, alltt, amssymb, xspace}
\usepackage{wrapfig}
\usepackage{fancyhdr}
\usepackage{url}
\usepackage{verbatim}
\usepackage{fancyvrb}
\usepackage{adjustbox}
\usepackage{listings}
\usepackage{color}
\usepackage{subfigure}
\usepackage{cite}
\usepackage{sidecap}
\usepackage{pifont}
\usepackage{mdframed}
\usepackage{textcomp}
\usepackage{enumitem}


% Horizontal alignment
\topmargin      -0.50in  % distance to headers
\oddsidemargin  0.0in
\evensidemargin 0.0in
\textwidth      6.5in
\textheight     8.9in 

\newcommand{\todo}[1]{
\vspace{0.1in}
\fbox{\parbox{6in}{TODO: #1}}
\vspace{0.1in}
}


\newcommand{\unix}{{\tt Unix}\xspace}
\newcommand{\linux}{{\tt Linux}\xspace}
\newcommand{\minix}{{\tt Minix}\xspace}
\newcommand{\ubuntu}{{\tt Ubuntu}\xspace}
\newcommand{\setuid}{{\tt Set-UID}\xspace}
\newcommand{\openssl} {\texttt{openssl}}


\pagestyle{fancy}
\lhead{\bfseries SEED Labs}
\chead{}
\rhead{\small \thepage}
\lfoot{}
\cfoot{}
\rfoot{}


\definecolor{dkgreen}{rgb}{0,0.6,0}
\definecolor{gray}{rgb}{0.5,0.5,0.5}
\definecolor{mauve}{rgb}{0.58,0,0.82}
\definecolor{lightgray}{gray}{0.90}


\lstset{%
  frame=none,
  language=,
  backgroundcolor=\color{lightgray},
  aboveskip=3mm,
  belowskip=3mm,
  showstringspaces=false,
%  columns=flexible,
  basicstyle={\small\ttfamily},
  numbers=none,
  numberstyle=\tiny\color{gray},
  keywordstyle=\color{blue},
  commentstyle=\color{dkgreen},
  stringstyle=\color{mauve},
  breaklines=true,
  breakatwhitespace=true,
  tabsize=3,
  columns=fullflexible,
  keepspaces=true,
  escapeinside={(*@}{@*)}
}

\newcommand{\newnote}[1]{
\vspace{0.1in}
\noindent
\fbox{\parbox{1.0\textwidth}{\textbf{Note:} #1}}
%\vspace{0.1in}
}


%% Submission
\newcommand{\seedsubmission}{You need to submit a detailed lab report, with screenshots,
to describe what you have done and what you have observed.
You also need to provide explanation
to the observations that are interesting or surprising.
Please also list the important code snippets followed by
explanation. Simply attaching code without any explanation will not
receive credits.}

%% Book
\newcommand{\seedbook}{\textit{Computer \& Internet Security: A Hands-on Approach}, 2nd
Edition, by Wenliang Du. See details at \url{https://www.handsonsecurity.net}.}

%% Videos
\newcommand{\seedisvideo}{\textit{Internet Security: A Hands-on Approach},
by Wenliang Du. See details at \url{https://www.handsonsecurity.net/video.html}.}

\newcommand{\seedcsvideo}{\textit{Computer Security: A Hands-on Approach},
by Wenliang Du. See details at \url{https://www.handsonsecurity.net/video.html}.}

%% Lab Environment
\newcommand{\seedenvironment}{This lab has been tested on our pre-built
Ubuntu 16.04 VM, which can be downloaded from the SEED website.}






\newcommand{\seedlabcopyright}[1]{
\vspace{0.1in}
\fbox{\parbox{6in}{\small Copyright \copyright\ {#1}\ \ by Wenliang Du.\\
      This work is licensed under a Creative Commons
      Attribution-NonCommercial-ShareAlike 4.0 International License.
      If you remix, transform, or build upon the material, 
      this copyright notice must be left intact, or reproduced in a way that is reasonable to
      the medium in which the work is being re-published.}}
\vspace{0.1in}
}






\lhead{\bfseries SEED Labs -- SQL Injection Attack Lab}

\newcommand{\sqlFigs}{./Figs}


\begin{document}


\begin{center}
{\LARGE SQL Injection Attack Lab }
\end{center}

\seedlabcopyright{2006 - 2016}


% *******************************************
% SECTION
% ******************************************* 
\section{Overview}

SQL injection is a code injection technique that exploits the 
vulnerabilities in the interface between web applications and 
database servers. The vulnerability is present when user's inputs 
are not correctly checked within the web applications 
before being sent to the back-end database servers.

Many web applications take inputs from users, and then use these
inputs to construct SQL queries, so they can get information from the database.
Web applications also use SQL queries to store information in
the database. These are common practices in the development of web applications.
When SQL queries are not carefully constructed, 
SQL injection vulnerabilities can occur. 
SQL injection is one of the most common 
attacks on web applications.


In this lab, we have created a web application that is vulnerable to the SQL injection attack. 
Our web application includes the common mistakes made by many web developers. 
Students' goal is to find ways to exploit the SQL injection vulnerabilities,
demonstrate the damage that can be achieved by the attack, 
and master the techniques that can help defend against such type of attacks.
This lab covers the following topics:

\begin{itemize}[noitemsep]
\item SQL statement: \texttt{SELECT} and \texttt{UPDATE} statements
\item SQL injection
\item Prepared statement
\end{itemize}



\paragraph{Readings.}
Detailed coverage of the SQL injection can be found in the following:

\begin{itemize}
\item Chapter 12 of the SEED Book, \seedbook
\end{itemize}

\paragraph{Lab Environment.} \seedenvironmentAB




% *******************************************
% SECTION
% ******************************************* 
\section{Lab Environment}


We have developed a web application for this lab. The folder where the 
application is installed and the URL to access this web application are 
described in the following: 

\begin{lstlisting}
   URL:    http://www.SEEDLabSQLInjection.com
   Folder: /var/www/SQLInjection/
\end{lstlisting}
 


%%%%%%%%%%%%%%%%%%%%%%%%%%%%%%%%%%%%
\newcommand{\urlorurls}{URL }
\newcommand{\urlisorurlsare}{URL is }


The above \urlisorurlsare is only accessible from inside of the virtual machine, because we
have modified the \texttt{/etc/hosts} file to map the domain
name of each URL to the virtual machine's local IP
address ({\tt 127.0.0.1}).
You may map any domain name to a particular IP address using
\texttt{/etc/hosts}. For example, you can map
\url{http://www.example.com} to the local IP address by appending the
following entry to \texttt{/etc/hosts}:

\begin{lstlisting}[backgroundcolor=]
   127.0.0.1     www.example.com
\end{lstlisting}

If your web server and browser are running on two different machines, you need
to modify \texttt{/etc/hosts} on the browser's machine accordingly to map these
domain names to the web server's IP address, not to {\tt 127.0.0.1}.




\paragraph{Apache Configuration.}
In our pre-built VM image, we used Apache server to host all the web
sites used in the lab. The name-based virtual hosting feature in
Apache could be used to host several web sites (or URLs) on the same
machine. A configuration file named \texttt{000-default.conf} in the directory
\url{"/etc/apache2/sites-available"} contains the necessary directives for the
configuration:

Inside the configuration file, each web site has a {\tt VirtualHost} block
that specifies the URL for the web site and directory in the file system
that contains the sources for the web site. The following examples show how
to configure a website with URL \url{http://www.example1.com} and another
website with URL \url{http://www.example2.com}:

\begin{lstlisting}
<VirtualHost *>
    ServerName http://www.example1.com
    DocumentRoot /var/www/Example_1/
</VirtualHost>

<VirtualHost *>
    ServerName http://www.example2.com
    DocumentRoot /var/www/Example_2/
</VirtualHost>
\end{lstlisting}


You may modify the web application by accessing the source in the
mentioned directories. For example, with the above configuration,
the web application \url{http://www.example1.com} can be changed by modifying
the sources in the \url{/var/www/Example_1/} directory. After a change is
made to the configuration, the Apache server needs to be restarted. See the
following command:

\begin{lstlisting}[backgroundcolor=]
   $ sudo service apache2 start
\end{lstlisting}


%%%%%%%%%%%%%%%%%%%%%%%%%%%%%%%%%%%%



% *******************************************
% SECTION
% ******************************************* 
\section{Lab Tasks}

We have created a web application, and host it at \url{www.SEEDLabSQLInjection.com}. 
This web application is a simple employee 
management application. 
Employees can view and update their personal information 
in the database through this web application. 
There are mainly two roles in this web application: 
{\tt Administrator} is a privilege role and can manage each individual
employees' profile information;
{\tt Employee} is a normal role and can view or update his/her own profile 
information. All employee information is described in the following table.


\vspace{0.1in}
\begin{table}[h]
\begin{adjustbox}{max width=\textwidth}
\begin{tabular}{|l|l|l|l|l|l|l|l|l|l|l|}
\hline
Name 	& Employee ID   & Password	&Salary		&Birthday
&SSN		&Nickname	&Email		&Address	&Phone\# \\
\hline
Admin 	& 99999         & seedadmin	&400000		&3/5		&43254314	&		&		&		&\\
Alice 	& 10000 	& seedalice	&20000		&9/20		&10211002	&		&		&		&\\
Boby 	& 20000         & seedboby 	&50000		&4/20		&10213352	&		&		&		&\\
Ryan    & 30000         & seedryan	&90000		&4/10		&32193525	&		&		&		&\\
Samy 	& 40000		& seedsamy	&40000		&1/11		&32111111 	&		&		&		&\\
Ted     & 50000		& seedted	&110000		&11/3		&24343244	&		&		&		&\\
\hline
\end{tabular}
\end{adjustbox}
\end{table}





% -------------------------------------------
% SUBSECTION
% ------------------------------------------- 
\subsection{Task 1: Get Familiar with SQL Statements}
\label{ssec:MySQLConsole}


The objective of this task is to get familiar with SQL commands by playing
with the provided database. We have created a database called {\tt Users},
which contains a table called {\tt credential}; the table stores
the personal information (e.g. eid, password, salary, ssn, etc.) of every
employee. In this task, you need to play with the database to get familiar
with SQL queries.


MySQL is an open-source relational database management system. 
We have already setup MySQL in our SEEDUbuntu VM image. 
The user name is {\tt root} and password is {\tt seedubuntu}.  
Please login to MySQL console using the following command:
	
\begin{lstlisting}[backgroundcolor=]
$ mysql -u root -pseedubuntu 
\end{lstlisting}


After login, you can create new database or load an existing
one. As we have already created the {\tt Users} database for you, you just
need to load this existing database using the following command: 

\begin{lstlisting}[backgroundcolor=]
  mysql> use Users;
\end{lstlisting}


To show what tables are there in the {\tt Users} database, 
you can use the following command to print out all the tables of the
selected database.

\begin{lstlisting}[backgroundcolor=]
  mysql> show tables;
\end{lstlisting}


After running the commands above, you need to use 
a SQL command to print all the profile information of
the employee {\tt Alice}. Please provide the screenshot of your results.





% -------------------------------------------
% SUBSECTION
% ------------------------------------------- 
\subsection{Task 2: SQL Injection Attack on SELECT Statement} 


SQL injection is basically a technique through
which attackers can execute their own malicious SQL statements generally
referred as malicious payload. Through the malicious SQL statements, 
attackers can steal information from the victim database; even worse,
they may be able to make changes to the database. Our employee
management web application has SQL injection vulnerabilities, which mimic 
the mistakes frequently made by developers. 

We will use the login page from \url{www.SEEDLabSQLInjection.com}
for this task. The login page is shown in Figure~\ref{sql:fig:login}. 
It asks users to provide a user name and a password.
The web application authenticate users based on these two pieces 
of data, so only employees who know their 
passwords are allowed to log in.
Your job, as an attacker, is to log into the web application without knowing
any employee's credential. 


\begin{figure}[htb]
\begin{center}
\includegraphics[width=0.7\textwidth]{\sqlFigs/login.jpg}
\end{center}
\caption{The Login page}
\label{sql:fig:login}
\end{figure}
 

To help you started with this task, we explain how authentication
is implemented in the web application. The PHP code 
\texttt{unsafe\_home.php}, located in the \url{/var/www/SQLInjection} directory, 
is used to conduct user authentication.
The following code snippet show how users are authenticated. 

\begin{lstlisting}
$input_uname = $_GET['username'];
$input_pwd = $_GET['Password'];
$hashed_pwd = sha1($input_pwd);
...
$sql = "SELECT id, name, eid, salary, birth, ssn, address, email, 
               nickname, Password
        FROM credential
        WHERE name= '$input_uname' and Password='$hashed_pwd'";
$result = $conn -> query($sql);

// The following is Pseudo Code 
if(id != NULL) {
  if(name=='admin') {
     return All employees information;
  } else if (name !=NULL){
    return employee information;
  }
} else {
  Authentication Fails;
}
\end{lstlisting}

The above SQL statement selects personal employee information such as id,
name, salary, ssn etc from the {\tt credential} table. The SQL statement
uses two variables \texttt{input\_uname} and \texttt{hashed\_pwd},
where \texttt{input\_uname} holds the string typed by 
users in the username field of the login page, while 
\texttt{hashed\_pwd} holds the \texttt{sha1} hash of the password typed by
the user. The program checks whether 
any record matches with the provided username and password; if there is a match,
the user is successfully authenticated, and is given the corresponding 
employee information. If there is no match, the authentication
fails. 



\begin{itemize}
\item {\bf Task 2.1: SQL Injection Attack from webpage}.
Your task is to log into the web application as 
the administrator from the login page, so you can see the information of
all the employees. We assume that you do know the administrator's account name
which is {\tt admin}, but you do not the password. You need to decide
what to type in the \texttt{Username} and \texttt{Password} fields to
succeed in the attack.
	

\item {\bf Task 2.2: SQL Injection Attack from command line}.  
Your task is to repeat Task 2.1, but you need to do it without 
using the webpage. You can use command line tools, such as 
\texttt{curl}, which can send HTTP requests. 
One thing that is worth mentioning is that if you
want to include multiple parameters in HTTP requests, you need to put the
URL and the parameters between a pair of single quotes; otherwise, the 
special characters used to separate parameters (such as \texttt{\&}) will be
interpreted by the shell program, changing the meaning of the 
command. The following example shows how to send an HTTP GET request to our 
web application, with two parameters (\texttt{username} and 
\texttt{Password}) attached:

\begin{lstlisting}
$ curl 'www.SeedLabSQLInjection.com/index.php?username=alice&Password=111'
\end{lstlisting}

If you need to include special characters in the 
\texttt{username} or \texttt{Password} fields, you need to 
encode them properly, or they can change the meaning of your 
requests. If you want to include single quote in those fields,
you should use \texttt{\%27} instead; if you want to include
white space, you should use \texttt{\%20}. In this
task, you do need to handle HTTP encoding while sending
requests using \texttt{curl}.


\item {\bf Task 2.3: Append a new SQL statement}.  
In the above two attacks, we can only steal information from the database;
it will be better if we can modify the database using the same
vulnerability in the login page.  An idea is to use the SQL injection
attack to turn one SQL statement into two, with the second one being the
update or delete statement. In SQL, semicolon (;) is used to separate two SQL
statements. Please describe how you can use the login page to get the
server run two SQL statements. Try the attack to delete a record from the
database, and describe your observation. 


%To be frank, we are unable to update the database using the  second
%approach. This is because of a particular defense mechanism implemented in
%MySQL. In the lab report, you should show us what you have tried in order
%to modify the database. You should find out why the attack fails, what
%mechanism in MySQL has prevented such an attack. You may look up evidences
%(second-hand) from the Internet to support your conclusion. However, a
%first-hand evidence will get more points (use your own creativity to find
%out first-hand evidences). If in case you find ways to succeed in the
%attacks, you will be awarded bonus points.

\end{itemize}



% -------------------------------------------
% SUBSECTION
% ------------------------------------------- 
\subsection{Task 3: SQL Injection Attack on UPDATE Statement} 

If a SQL injection vulnerability happens to an UPDATE statement, the damage will be more
severe, because attackers can use the vulnerability to modify databases. 
In our Employee Management application, there is an Edit Profile page~(Figure~\ref{sql:fig:edit}) 
that allows employees to
update their profile information, including nickname, email, address, phone number, and
password. To go to this page, employees need to log in first. 


When employees update their information through the Edit Profile page, the
following SQL UPDATE query will be executed. The PHP code implemented in
{\tt unsafe\_edit\_backend.php} file is used to update employee's profile
information. The PHP file is located in the {\tt /var/www/SQLInjection}
directory.


\begin{lstlisting}
$hashed_pwd = sha1($input_pwd);
$sql = "UPDATE credential SET
	nickname='$input_nickname',
	email='$input_email',
	address='$input_address',
	Password='$hashed_pwd',
	PhoneNumber='$input_phonenumber'
	WHERE ID=$id;";
$conn->query($sql);
\end{lstlisting}
 

\begin{figure}[htb]
\begin{center}
  \includegraphics[width=0.6\textwidth]{\sqlFigs/editprofile.jpg}
\end{center}
\caption{The Edit-Profile page}
\label{sql:fig:edit}
\end{figure}
 


\begin{itemize}
\item {\bf Task 3.1: Modify your own salary}.  
As shown in the Edit Profile page,
employees can only update their nicknames, emails, addresses, phone numbers, and
passwords; they are not authorized to change their salaries.  
Assume that you (Alice) are a disgruntled employee, and your boss Boby did not 
increase your salary this year. You want to increase your own salary 
by exploiting the SQL injection vulnerability in
the Edit-Profile page. Please demonstrate how you can achieve that.
We assume that you do know that salaries are stored in 
a column called ’{\tt salary}’.


\item {\bf Task 3.2: Modify other people' salary}.
After increasing your own salary, you decide to punish your boss Boby. You want to reduce his
salary to 1 dollar. Please demonstrate how you can achieve that. 


\item {\bf Task 3.3: Modify other people' password}.
After changing Boby's salary, you are still disgruntled, so you
want to change Boby's password to something that you know, and then you can log into his account
and do further damage. Please demonstrate how you can achieve that.
You need to demonstrate that you can 
successfully log into Boby's account using the new
password.  One thing worth mentioning here is that the database stores the hash value of
passwords instead of the plaintext password string. You can again look at
the {\tt unsafe\_edit\_backend.php} code to see how password is being stored. It
uses SHA1 hash function to generate the hash value of password. 


\end{itemize}

To make sure your injection string does not contain any syntax error, 
you can test your injection string on MySQL console before launching the
real attack on our web application. 



\subsection{Task 4: Countermeasure --- Prepared Statement} 

The fundamental problem of the SQL injection vulnerability is the failure to
separate code from data. When constructing a SQL statement, the program
(e.g. PHP program) knows which part is data and which part is code.
Unfortunately, when the SQL statement is sent to the database, the boundary
has disappeared; the boundaries that the SQL interpreter sees may be
different from the original boundaries that was set by the developers.
To solve this problem, it is important to ensure that the view
of the boundaries are consistent in the server-side code and in the
database.  The most secure way is to use 
\textit{prepared statement}. 

\begin{figure}
\centering
\includegraphics[width=0.8\textwidth]{\sqlFigs/PreparedStatement.pdf}
\caption{Prepared Statement Workflow}
\label{sql:fig:preparedstatement}
\end{figure}


To understand how prepared statement prevents SQL injection, 
we need to understand what happens when SQL server receives a query. 
The high-level workflow of how queries are executed is shown in Figure~\ref{sql:fig:preparedstatement}.
In the compilation step, queries first go through the parsing and normalization phase, 
where a query is checked against the syntax and semantics. 
The next phase is the compilation phase where keywords~(e.g. SELECT, FROM, UPDATE, etc.) 
are converted into a format understandable to machines. 
Basically, in this phase, query is interpreted.
In the query optimization phase, the number of different plans are considered to 
execute the query,  out of which the best optimized plan is chosen. 
The chosen plan is store in the cache, so 
whenever the next query comes in, 
it will be checked against the content in the cache; if it's already present in the cache,
the parsing, compilation and query optimization phases will be skipped. 
The compiled query is then passed to the execution phase 
where it is actually executed.


Prepared statement comes into the picture after the compilation but before the execution step. 
A prepared statement will go through the compilation step, and be turned into
a pre-compiled query with empty placeholders for data. To run this pre-compiled query,
data need to be provided, but these data will not go through the compilation step; instead,
they are plugged directly into the pre-compiled query, and are sent to the execution engine.
Therefore, even if there is SQL code inside the data, without going through the compilation
step, the code will be simply treated as part of data, without any special meaning.  
This is how prepared statement prevents SQL injection attacks.


Here is an example of how to write a prepared statement in PHP.  We use a SELECT statment in
the following example.  We show how to use prepared statement to rewrite the code that is
vulnerable to SQL injection attacks.


\begin{lstlisting}
   $sql = "SELECT name, local, gender  
           FROM USER_TABLE 
           WHERE id = $id AND password ='$pwd' ";
   $result = $conn->query($sql)) 
\end{lstlisting}

The above code is vulnerable to SQL injection attacks. 
It can be rewritten to the following


\begin{lstlisting}
   $stmt = $conn->prepare("SELECT name, local, gender
                           FROM USER_TABLE 
                           WHERE id = ? and password = ? ");
   // Bind parameters to the query
   $stmt->bind_param("is", $id, $pwd);
   $stmt->execute();
   $stmt->bind_result($bind_name, $bind_local, $bind_gender);
   $stmt->fetch();
\end{lstlisting}


Using the prepared statement mechanism, we divide the process of sending
a SQL statement to the database into two steps.  
The first step is to only send the code part, i.e., a SQL statement without 
the actual the data. This is the prepare step. As we can see from the 
above code snippet, the actual data are replaced by question
marks (?).  After this step, we then send the data to the database using 
{\tt bind\_param()}.
The database will treat everything sent in this step only as 
data, not as code anymore. It binds the data to the corresponding
question marks of the prepared statement. 
In the {\tt bind\_param()} method, the first argument {\tt "is"} indicates
the types of the parameters: \texttt{"i"} means  
that the data in {\tt \$id} has the integer type,
and \texttt{"s" } means that the data in {\tt \$pwd} has the string type.


For this task, please use the prepared statement mechanism to 
fix the SQL injection vulnerabilities exploited by you in the previous tasks. 
Then, check whether you can still exploit the vulnerability or not. 


\section{Guidelines}
\label{sec:guidelines}

\paragraph{Test SQL Injection String.}
In real-world applications, it may be hard to check whether your SQL injection attack contains
any syntax error, because usually servers do not return this kind of error messages. 
To conduct your investigation, you can copy the SQL statement from php source code to the MySQL console. 
Assume you have the following SQL statement, and the injection string is {\tt ' or 1=1;\#}. 
\begin{verbatim}
SELECT * from credential 
	WHERE name='$name' and password='$pwd';
\end{verbatim} 
You can replace the value of {\tt \$name} with the
injection string and test it using the MySQL console. 
This approach can help you to construct a syntax-error 
free injection string before launching the real injection attack. 


% This part is removed 
\begin{comment}
\paragraph{Escaping Special Characters using magic\_quotes\_gpc}

You will get to know that SQL Injection attack is possible because of attacker can use some
special character to alter the existing SQL queries.  In the PHP code, if a data variable is a
string type, it needs to be enclosed within a pair of single quote (').  For example, in the
SQL query listed above, we have used {\tt name = `\$user'}.  The single quote symbol
surrounding {\tt \$user} basically “tries” to separate the data in the {\tt \$user} variable
from the code.  Unfortunately, this separation will fail if the content of {\tt \$user}
variable include any single quote.  Therefore, we need a mechanism to tell the database that
the single quote in {\tt \$user} should be treated as a part of the data, not like the special
character in SQL.  All we need to do is to add a backslash (\textbackslash) before the single
quote, which will prevent us to alter any existing SQL query.  PHP provides a mechanism to
automatically add a backslash before single-quote ('), double quote ("), backslash
(\textbackslash), and NULL characters.  If this mechanism is turned on, all of these characters
in the user inputs will be automatically escaped.  This mechanism is known as magical quotes
and generally refer by the value of {\tt magic\_quotes\_gpc}.

 
Please note that, magic\_quotes\_gpc feature has been DEPRECATE as of 5.3.0 and REMOVED as of
PHP 5.4.0.  The PHP version installed in SEEDUbuntu VM is 5.3.5, so you can still play with
this. The reasons why it is removed is described below:

\begin{itemize}
\item Portability: Assuming it to be on, or off, affects portability.
Most code has to use a function called {\tt get\_magic\_quotes\_gpc()} to check for this,
and code accordingly.

\item Performance and Inconvenience: Not all user inputs are used
for SQL queries, so mandatory escaping all data not only affects
performance, but also become annoying when some data are not
supposed to be escaped.
\end{itemize}

\end{comment}



% *******************************************
% SECTION
% ******************************************* 
\section{Submission}

\seedsubmission


%%%%%%%%%%%%%%%%%%%%%%%%%%%%%%%%%%%%%%%%%%%%%
\end{document}
%%%%%%%%%%%%%%%%%%%%%%%%%%%%%%%%%%%%%%%%%%%%%


% This part is no longer used 
\appendix

\section{Patching the Lab Environment}

In the environment setup section, we described how to run 
our patch script to set up the SQL injection lab environment. 
If you are interested to setting up the environment manually, 
this section provides step-by-step instructions. 

\subsection{Load data into database}

We need to load some existing data into MySQL. 
The following commands log into the MySQL database, create a database
called \texttt{Users}, and load the data from {\tt Users.sql} into 
the newly created database. 


\begin{lstlisting}[frame=single, caption={}, label=label]
$ mysql -u root -pseedubuntu
mysql> CREATE DATABASE Users;
mysql> quit
$ mysql -u root -pseedubuntu Users < Users.sql
\end{lstlisting}






\subsection{Set up the website}

Four steps are needed to set up the Employee Management web application
inside the VM.


\noindent
Step 1: Create a new folder \url{/var/www/SQLInjection} and copy all the 
php, html,c cs files to \url{/var/www/SQLInjection}. Please go to the
directory where you store the patch folder and
type the following commands. 
 
\begin{Verbatim}[frame=single]
$ sudo mkdir /var/www/SQLInjection
$ sudo cp *.css *.php *.html /var/www/SQLInjection
\end{Verbatim}

\noindent
Step 2: In the SEEDUbuntu VM, we use Apache to host all the web
sites used in the lab. The name-based virtual hosting feature in Apache
can be used to host several web sites (or URLs) on the same machine.  A
configuration file {\tt /etc/apache2/sites-available/default} contains the
neccessary directives for the configuration.
We can add a new url to {\tt /etc/apache2/sites-available/default} file as
follows: 
\begin{Verbatim}[frame=single]
<VirtualHost *>
    ServerName http://www.SeedLabSQLInjection.com
    DocumentRoot /var/www/SQLInjection/
</VirtualHost>
\end{Verbatim}


\noindent
Step 3: We need to modify the local DNS file {\tt /etc/hosts} to
provide an IP address (127.0.0.1)  for the host name
\url{www.SeedLabSQLInjection.com}.

\begin{Verbatim}[frame=single]
127.0.0.1	www.SeedLabSQLInjection.com
\end{Verbatim}


\noindent
Step 4: Restart the Apache server.

\begin{Verbatim}[frame=single]
$ sudo service apache2 restart
\end{Verbatim}







